\chapter{Metodologia}
Neste capítulo é apresentada a metodologia utilizada para o desenvolvimento do aplicativo. 
[EM DESENVOLVIMENTO até 31/03/23]

% Após o levantamento dos conceitos necessários para o desenvolvimento da aplicação e um estudo de aplicativos similares na literatura, o próximo passo foi um levantamento dos sistemas existente na CBMPA. Revisão bibliográfica: descrever as principais referências utilizadas para a construção da aplicação móvel, incluindo frameworks, bibliotecas e outras tecnologias relacionadas.

% Revisão bibliográfica: descrever as principais referências utilizadas para a construção da aplicação móvel, incluindo frameworks, bibliotecas e outras tecnologias relacionadas.

% Especificação de requisitos: descrever os requisitos funcionais e não funcionais da aplicação móvel, incluindo as funcionalidades que ela deve ter e as plataformas em que deve ser executada.

% Prototipagem: descrever o processo de prototipagem da aplicação móvel, incluindo a criação de wireframes, modelos de tela e fluxos de navegação.

% Desenvolvimento: descrever o processo de desenvolvimento da aplicação móvel, incluindo as ferramentas e tecnologias utilizadas, como a linguagem de programação, banco de dados e plataformas de desenvolvimento.

% Testes: descrever o processo de testes da aplicação móvel, incluindo os tipos de testes realizados, como testes funcionais e de desempenho.

% Implantação: descrever o processo de implantação da aplicação móvel, incluindo a distribuição nas lojas de aplicativos, atualizações e manutenção.

