%A análise e cálculo de atividades enzimáticas do solo é uma tarefa muito importante para tomada de decisões de estudantes e pesquisadores na área da agronomia, pois ela é fundamental para a produção de alimentos no Brasil e no mundo, já que a fertilidade do solo é indispensável para o crescimento das plantas e a produtividade dos cultivos. A análise do solo permite aos agricultores tomar decisões precisas sobre práticas de manejo, identificar problemas de contaminação e remediar a situação. Desta forma, é crucial possuir informações precisas sobre os recursos que se deseja avaliar. Acontece que, muitas vezes, as ferramentas disponíveis não oferecem flexibilidade suficiente para coletar dados, realizar cálculos e analisar os resultados de maneira ágil e simples. Sendo assim, o presente trabalho propõe um sistema para o gerenciamento de experimentos e resultados das análises das atividades enzimáticas do solo coletadas, denominado Enzitech. Para implementação de proposta, foram utilizadas técnicas de desenvolvimento ágeis, arquitetura limpa e código limpo, sendo escolhida os \acp{dm} como plataforma alvo, gerando um aplicativo que dá acesso a todo o sistema, garantindo a mobilidade do usuário, o qual conseguirá criar e gerenciar seus experimentos independente de onde estiver.

Diante da projeção do crescimento da população mundial, a \ac{qs} é um dos fatores fundamentais para o aumento da produtividade agrícola e para a redução da escassez global de alimentos. Nesse contexto, é fundamental entender a \ac{qs}, suas características físicas, químicas e biológicas, bem como as interações entre esses fatores, permitindo o desenvolvimento de técnicas mais eficientes e sustentáveis de manejo do solo. Diversos métodos são utilizados para avaliar a \ac{qs}, entre eles, o cálculo de \acp{ae} do solo é uma importante ferramenta para avaliar a qualidade e saúde do solo, uma vez que a \acp{ae} está diretamente relacionada com a quantidade e diversidade de microrganismos presentes no solo. Desta forma, é de fundamental importância a disponibilidade de informação precisa e também de sistemas computacionais para cálculo das \acp{ae} e para auxílio na tomada de decisão. Dessa maneira, este trabalho propõe um sistema para o gerenciamento de experimentos e resultados das análises das \acp{ae} do solo. Esta abordagem foi implementada através de conceitos como métodos ágeis e com arquitetura limpa, contemplando inicialmente o cálculo das enzimas fosfatase ácida e alcalina, a $\beta$-glucosidase, a urease e a protease. Esta ferramenta auxilia pesquisadores, laboratórios e profissionais da área, que podem realizar experimentos de solo com embasamento científico de forma prática e precisa, aperfeiçoando as práticas de manejo, identificação de problemas de contaminação, tratamento e remediação. 

\begin{keywords}
Cálculo de atividades enzimáticas do solo, Engenharia de Software, Desenvolvimento Mobile, Metodologias Ágeis, Arquitetura Limpa, Flutter.
\end{keywords}