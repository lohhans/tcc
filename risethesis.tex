%% RiSE Latex Template - version 0.5
%%
%% RiSE's latex template for thesis and dissertations
%% http://risetemplate.sourceforge.net
%%
%% (c) 2012 Yguaratã Cerqueira Cavalcanti (yguarata@gmail.com)
%%          Vinicius Cardoso Garcia (vinicius.garcia@gmail.com)
%%
%% This document was initially based on UFPEThesis template, from Paulo Gustavo
%% S. Fonseca.
%%
%% ACKNOWLEDGEMENTS
%%
%% We would like to thanks the RiSE's researchers community, the 
%% students from Federal University of Pernambuco, and other users that have
%% been contributing to this projects with comments and patches.
%%
%% GENERAL INSTRUCTIONS
%%
%% We strongly recommend you to compile your documents using pdflatex command.
%% It is also recommend use the texlipse plugin for Eclipse to edit your documents.
%%
%% Options for \documentclass command:
%%         * Idiom
%%           pt   - Portguese (default)
%%           en   - English
%%
%%         * Text type
%%           bsc  - B.Sc. Thesis
%%           msc  - M.Sc. Thesis (default)
%%           qual - PHD qualification (not tested yet)
%%           prop - PHD proposal (not tested yet)
%%           phd  - PHD thesis
%%
%%         * Media
%%           scr  - to eletronic version (PDF) / see the users guide
%%
%%         * Pagination
%%           oneside - unique face press
%%           twoside - two faces press
%%
%%		   * Line spacing
%%           singlespacing  - the same as using \linespread{1}
%%           onehalfspacing - the same as using \linespread{1.3}
%%           doublespacing  - the same as using \linespread{1.6}
%%
%% Reference commands. Use the following commands to make references in your
%% text:
%%          \figref  -- for Figure reference
%%          \tabref  -- for Table reference
%%          \eqnref  -- for equation reference
%%          \chapref -- for chapter reference
%%          \secref  -- for section reference
%%          \appref  -- for appendix reference
%%          \axiref  -- for axiom reference
%%          \conjref -- for conjecture reference
%%          \defref  -- for definition reference
%%          \lemref  -- for lemma reference
%%          \theoref -- for theorem reference
%%          \corref  -- for corollary reference
%%          \propref -- for proprosition reference
%%          \pgref   -- for page reference
%%
%%          Example: See \chapref{chap:introduction}. It will produce 
%%                   'See Chapter 1', in case of English language.

\documentclass[pt,oneside,onehalfspacing,bsc]{risethesis}

\usepackage[english]{babel}
\usepackage{colortbl}
\usepackage{color}
\usepackage[table]{xcolor}
\usepackage{microtype}
\usepackage{bibentry}
\usepackage{subfigure}
\usepackage{multirow}
\usepackage{rotating}
\usepackage{booktabs}
\usepackage{pdfpages}
\usepackage{caption}
\usepackage{lipsum}

\captionsetup[table]{position=top,justification=centering,width=.85\textwidth,labelfont=bf,font=small}
\captionsetup[lstlisting]{position=top,justification=centering,width=.85\textwidth,labelfont=bf,font=small}
\captionsetup[figure]{position=bottom,justification=centering,width=.85\textwidth,labelfont=bf,font=small}

%% Change the following pdf author attribute name to your name.
\usepackage[linkcolor=black,
            citecolor=blue,
            urlcolor=black,
            colorlinks,
            pdfpagelabels,
            pdftitle={Rise Thesis Template (ABNT)},
            pdfauthor={Rise Thesis Template (ABNT)}]{hyperref}

\address{Garanhuns-PE}

\universitypt{Universidade Federal do Agreste de Pernambuco}
\universityen{Federal University of the Agreste of Pernambuco}

% \departmentpt{Centro de Informática}
% \departmenten{Center for Informatics}

\programpt{Graduação em Ciência da Computação}
\programen{Graduate in Computer Science}

\majorfieldpt{Ciência da Computação}
\majorfielden{Computer Science}

\title{Enzitech: Um aplicativo para experimento e análise de solo}

\date{2022}

\author{Armstrong Lohãns de Melo Gomes Quintino}
\adviser{Rodrigo Gusmão de Carvalho Rocha}

% Macros (defines your own macros here, if needed)
\def\x{\checkmark}

\begin{document}

\frontmatter

\frontpage

\presentationpage

\begin{fichacatalografica}
	\FakeFichaCatalografica % Comment this line when you have the correct file
%     \includepdf{fig_ficha_catalografica.pdf} % Uncomment this
\end{fichacatalografica}

\banca

\begin{dedicatory}
% TODO: Mudar
Eu dedico este trabalho a todos os meus familiares, amigos e colegas que nunca deixaram de acreditar em mim.
\end{dedicatory}

\acknowledgements
% TODO: Mudar
Primeiramente a Deus, por ter me concedido o conhecimento, a saúde e a força de vontade necessários para alcançar até aqui.

Aos meus pais, Ricarte Gomes Quintino e Ivanize Alves de Melo Quintino, que batalharam duro, nunca mediram esforços para minha educação e sempre me ensinaram a lutar para atingir meus objetivos, meus primeiros amigos, que me ensinaram valores que levarei para toda a vida. Essa conquista é nossa, amo vocês!

A minha irmã, Roane Lohaynne, a qual sempre foi um espelho como irmã mais velha, que sempre esteve ao meu lado. Você é uma fonte de inspiração e admiração para mim, e tenho muito orgulho de ser seu irmão. Dedico este trabalho a você com todo o meu amor e gratidão.

A minha namorada, Ana Beatriz, que me apoiou em todas as fases dessa jornada. Agradeço por ter estado ao meu lado em todos os momentos, sempre acreditando e me incentivando a buscar o meu melhor. Pude iniciar e concluir esta graduação com você ao meu lado, e espero que possamos continuar juntos nos próximos ciclos da vida. Eu te amo muito, você é um exemplo de amor e companheirismo!

Aos meus queridos amigos do OBDGUS e amigos agregados, em especial Alessandro, Andrei, Cézar, Elyson, Erik e Morgan, por estarem juntos comigo sempre que precisei. Todos vocês participaram de algo na minha graduação — me apoiaram com todas as dificuldades que encontrei no caminho e nunca desacreditaram — mas acima de tudo, participaram da minha vida, como grandes amigos que levarei comigo para sempre. Que venham mais peladas, acampamentos, brejas e partidas de CS! \textit{Tamo junto}!

Aos meus amigos de turma e do nosso grupo "O Núcleo", formado lá no começo do curso, em especial meus amigos Anderson, André, Antônio, Baltazar, Daniel, Igor, Júnior (CJ), Laisy, Luis, Machado, Saú, Valdir e Victor, que contribuíram com minha formação, e apesar de seguirmos um pouco distantes, sei que posso contar com muitos, obrigado pelos anos de conversas e conhecimentos trocados. Carrego todos vocês no peito.

Aos demais amigos e colegas que fiz na graduação, aqui em especial Matheus Noronha e Weverton Cintra, vocês me ajudaram muito nessa fase final do curso, devo muito a vocês, agradeço demais pelas horas disponibilizadas em meio a tanta coisa, sem a ajuda de vocês isso aqui não seria possível.

A todos os professores que me fizeram ser um ser humano melhor, desde o ensino infantil até aqui, principalmente aos professores que fazem parte da nossa universidade e que me instruíram de forma excelente para os desafios da vida profissional.

E claro, agradeço ao professor orientador, Dr. Rodrigo Gusmão de Carvalho Rocha por todo o apoio ao longo do curso, em especial, nesta reta final, pegando na minha mão e me ajudando a alcançar este feito, me auxiliando com todos os contratempos enfrentados.

Para finalizar, a todos os meus colegas e amigos, que não mencionei aqui, mas que de alguma forma participaram deste ciclo maravilhoso. Obrigado!


\begin{epigraph}[]{ Mewtwo}
``As circunstâncias do nascimento de alguém são irrelevantes; é o que você faz com seu dom da vida que determina quem você é.``
\end{epigraph}

\resumo
% Escreva seu resumo no arquivo resumo.tex
\lipsum[1]

\begin{keywords}
Cálculo de atividades enzimáticas do solo, Engenharia de Software, Desenvolvimento Mobile, Arquitetura Limpa, Flutter
\end{keywords}

\abstract
% Write your abstract in a file called abstract.tex
\lipsum[1]

\begin{keywords}
Calculation of soil enzymatic activities, Software Engineering, Mobile Development, Clean Architecture, Flutter
\end{keywords}

% List of figures
\listoffigures

% List of tables
\listoftables

% List of acronyms
% Acronyms manual: http://linorg.usp.br/CTAN/macros/latex/contrib/acronym/acronym.pdf
\listofacronyms
\begin{acronym}[ACRONYM] 
% Change the word ACRONYM above to change the acronym column width.
% The column width is equals to the width of the word that you put.
% Read the manual about acronym package for more examples:
%   http://linorg.usp.br/CTAN/macros/latex/contrib/acronym/acronym.pdf
\acro{ae}[AE]{Atividade Enzimática}
\acrodefplural{ae}{Atividades Enzimáticas}
\acro{am}[AM]{Aprendizado de Máquina}
\acro{app}[APP]{Aplicativo}
\acro{api}[API]{\textit{Application Programming Interface}}
% \acro{apm}[APM]{Aplicativo Móvel}
% \acrodefplural{apm}{Aplicativos Móveis}
\acro{ba}[BA]{Bacharelado em Agronomia}
\acro{bcc}[BCC]{Bacharelado em Ciência da Computação}
\acro{cagr}[CAGR]{Taxa de Crescimento Anual Composta}
\acro{crud}[CRUD]{\textit{Create, Read, Update and Delete}}
\acro{dto}[DTO]{\textit{Data Transfer Object}}
\acro{dm}[DM]{Dispositivo Móvel}
\acrodefplural{dm}{Dispositivos Móveis}
\acro{fao}[FAO]{Organização das Nações Unidas para Agricultura e Alimentação}
\acro{http}[HTTP]{\textit{Hypertext Transfer Protocol}}
\acro{ia}[IA]{Inteligência Artificial}
\acro{ide}[IDE]{\textit{Integrated Development Environment}}
\acro{json}[JSON]{\textit{JavaScript Object Notation}}
\acro{mvp}[MVP]{Produto Viável Mínimo}
\acro{mvvm}[MVVM]{\textit{Model–view–viewmodel}}
\acro{ndc}[NDC]{Contribuição Nacionalmente Determinada}
\acro{onu}[ONU]{Organização das Nações Unidada}
\acro{pc}[PC]{Computador Pessoal}
\acrodefplural{pc}{Computadores Pessoais}
\acro{planoabc}[Plano ABC]{Plano de Agricultura de Baixa Emissão de Carbono}
\acro{rest}[REST]{\textit{Representational State Transfer}}
\acro{po}[PO]{\textit{Product Owner}}
\acro{qs}[QS]{Qualidade do Solo}
\acro{sdk}[SDK]{\textit{Software Development Kit}}
\acro{so}[SO]{Sistema Operacional}
\acrodefplural{so}{Sistemas Operacionais}
\acro{tcc}[TCC]{Trabalho de Conclusão de Curso}
\acro{ufape}[UFAPE]{Universidade Federal do Agreste de Pernambuco}
\acro{ui}[UI]{\textit{User Interface}}
\acro{uml}[UML]{\textit{Unified Modeling Language}}
\acro{url}[URL]{\textit{Uniform Resource Locator}}
\acro{xml}[XML]{\textit{Extensible Markup Language}}
% \acro{soho}[SOHO]{Small Home Office}
% \acro{api}[API]{Application Programming Interface}
% \acro{arima}[ARIMA]{Auto-Regressive Integrated Moving Average}
% \acro{brn}[BRN]{Bug Report Network}
% \acro{bts}[BTS]{Bug Triage System}
% \acro{cas}[CAS]{Context-Aware Systems}
% \acro{ccb}[CCB]{Change Control Board}
\acro{cr}[CR]{Change Request}
% \acro{cvs}[CVS]{Concurrent Version System}
% \acro{es}[ES]{Expert System}
% \acro{floss}[FLOSS]{Free/Libre Open Source Software}
% \acro{glr}[GLR]{Generalized Linear Regression}
% \acro{gqm}[GQM]{Goal Question Metric}
% \acro{html}[HTML]{HyperText Markup Language}
% \acro{ir}[IR]{Information Retrieval}
% \acro{irt}[IRT]{Recôncavo Institute of Technology}
% \acro{jdt}[JDT]{Jazz Duplicate Finder}
% \acro{lda}[LDA]{Latent Dirichlet Allocation}
% \acro{loc}[LOC]{Lines of Code}
% \acro{lsi}[LSI]{Latent Semantic Indexing}
% \acro{ms}[MS]{Mapping Study}
% \acro{msr}[MSR]{Mining Software Repositories}
% \acro{nlp}[NLP]{Natural Language Processing}
% \acro{promise}[PROMISE]{Predictive Models in Software Engineering}
% \acro{rbes}[RBES]{Rule-Based Expert System}
% \acro{rhel}[RHEL]{RedHat Enterprise Linux}
% \acro{saas}[SaaS]{Software as a Service}
% \acro{scm}[SCM]{Software Configuration Management}
% \acro{serpro}[SERPRO]{Brazilian Federal Organization for Data Processing}
% \acro{slr}[SLR]{Stepwise Linear Regression}
% \acro{slr}[SLR]{Systematic Literature Review}
% \acro{svd}[SVD]{Singular Value Decomposition}
% \acro{svm}[SVM]{Support Vector Machine}
% \acro{svn}[SVN]{Subversion}
% \acro{tfidf}[TF-IDF]{Term Frequency-Inverse Document Frequency}
% \acro{vsm}[VSM]{Vector Space Model}
% \acro{xp}[XP]{Extreming Programming}
\end{acronym}

% Summary (tables of contents)
\tableofcontents

\mainmatter

\chapter{Introdução}
\label{chp:introduction}
A eficiente e incessante evolução da tecnologia traz consigo a grande oportunidade — como também a necessidade — de tornar as tarefas cotidianas dos seres humanos cada vez mais triviais e informatizadas, ou seja, realizar uma tarefa de forma que exija o menor esforço possível, dessa forma, algo que requisitaria tempo e atenção de alguém vem a se tornar uma tarefa simples para um computador.

As aplicações móveis, habitualmente conhecidas por \acp{app}, são exemplos claros e amplamente difundidos atualmente, visto que cada vez mais tarefas, desde o envio de mensagens até o pagamento de contas, podem ser realizadas nas palmas das nossas mãos, basicamente utilizando um \textit{smartphone} com acesso à Internet. O que há pouco mais de uma década era visualizado como uma atividade exclusiva tecnologicamente de um \ac{pc}, fosse através de uma aplicação \textit{web} ou \textit{desktop}, hoje é facilmente feito por soluções adaptadas ou similares para as telas dos celulares, tornando-se uma atividade corriqueira na vida da maioria das pessoas.

Em outra esfera, a análise de \acp{ae} do solo é um importante indicador da qualidade e saúde do solo, pois essas atividades estão diretamente relacionadas com a disponibilidade de nutrientes e a capacidade do solo em sustentar a vida vegetal. Porém, as técnicas convencionais para a medição dessas atividades são trabalhosas, caras e requerem equipamentos sofisticados.

Segundo \cite{nascimento2017impacto}, o uso de tecnologias nas instituições de pesquisa científica em agronomia tem sido amplamente adotado nas últimas décadas, com o objetivo de aumentar a eficiência e a precisão dos experimentos, bem como de facilitar a coleta e análise de dados. A implementação de sistemas informatizados e a utilização de equipamentos avançados de análise, como espectrômetros e microscópios eletrônicos, são exemplos dessas tecnologias que vêm sendo empregadas em estudos agronômicos.

Nesse contexto, o desenvolvimento de um aplicativo móvel para a realização de experimentos de análise e cálculo de \acp{ae} do solo pode ser uma solução eficaz e prática para essa demanda. O objetivo deste trabalho é apresentar o desenvolvimento de um aplicativo móvel que permite a realização destes experimentos, com o intuito de facilitar e tornar mais acessível a análise dessas atividades, buscando harmonizar o crescimento econômico, a equidade social e a preservação ambiental, visando à conservação do meio ambiente e à garantia do bem-estar humano a longo prazo.

\section[Motivação]{Motivação}
Os experimentos do solo com cálculo de \acp{ae} são uma técnica utilizada em estudos de ciência do solo para avaliar a saúde e qualidade do solo, estes experimentos apresentam alguns desafios e lacunas, como a dificuldade na padronização dos métodos de análise, a falta de informações sobre a relação entre as \acp{ae} e a fertilidade do solo e a dificuldade em avaliar a dinâmica temporal das \acp{ae}. Além disso, há pouco conhecimento sobre as \acp{ae} em diferentes tipos de solos e em diferentes regiões do mundo, e a identificação de microrganismos e enzimas específicas pode ser desafiadora. Esses desafios podem afetar a interpretação dos resultados dos experimentos e limitar a aplicação das técnicas em diferentes contextos \cite{tabatabai1994soil}.

Este tema é importante para a comunidade acadêmica e para a sociedade em geral, pois permite avaliar a saúde e qualidade do solo, identificar microrganismos e enzimas benéficos para o solo e para as plantas, promovendo práticas agrícolas mais sustentáveis e eficientes, além de monitorar a qualidade do solo em áreas urbanas e industriais. A pesquisa e desenvolvimento do sistema proposto para essa área contribui para o desenvolvimento sustentável e para a conservação do meio ambiente.
 
\section{Objetivo Geral}\label{sec:objetivo_geral}

Este trabalho tem como objetivo detalhar a solução e o processo de desenvolvimento de uma aplicação para dispositivos móveis que informatize o método de fazer experimentos e análises do solo, aplicativo este desenvolvido usando padrões arquiteturais limpos, escaláveis, manuteníveis e testáveis.

\section{Objetivos específicos}\label{sec:objetivo_específico}
Com base no objetivo geral, correspondem os objetivos específicos indicados a seguir:

\begin{itemize}
    \item Implementar um sistema, acessível via aplicativo móvel, que permite criar e gerenciar experimentos que realizem cálculos das \acp{ae} do solo;
    \item Gerar resultados e planilhas de forma automática;
    \item Possibilitar a análise dos resultados;
    \item Proporcionar uma experiência fluida de navegação no aplicativo, ao utilizar boas práticas de desenvolvimento de \textit{software}.
\end{itemize}

\section{Organização do texto}

O presente trabalho foi assim constituído:
\begin{itemize}
    \item O Capítulo \ref{ch:teoria} discorre sobre conceitos básicos dos principais fundamentos adotados neste trabalho e relacionados ao tema científico, em que são abordado os temas de \acp{ae} do solo, dispositivos móveis, desenvolvimento de aplicativos móveis e arquitetura limpa;
    \item No Capítulo \ref{ch:metodos} é realizada uma apresentação dos métodos utilizados no desenvolvimento da solução, bem como os requisitos do sistema, tecnologias e ferramentas envolvidas;
    \item No Capítulo \ref{ch:proposta}, são apresentados os processos do desenvolvimento, arquiteturas utilizadas, informações sobre os testes e implantação do aplicativo e do sistema em si;
    \item No Capítulo \ref{ch:resultados}, é apresentado a versão final do aplicativo desenvolvido, o fluxo de navegação e como todo o sistema se comporta;
    \item No Capítulo \ref{ch:conclusao}, são apresentadas as considerações finais, principais contribuições, dificuldades encontradas e trabalhos futuros.
\end{itemize}

\chapter{Fundamentação Teórica}

\section{Mobilidade}\label{sec:mobilidade}
Segundo \citet{b2004mobile}, um sistema de computação móvel é um sistema que pode ser facilmente movido fisicamente ou cuja funcionalidade pode ser usada durante o movimento. Como esses sistemas fornecem essa mobilidade, essas funcionalidades adicionadas são a razão para caracterizar separadamente os sistemas de computação móvel, eles geralmente oferecem capacidades e recursos não encontrados em sistemas normais, como: 
 \begin{itemize}
   \item Armazenamento de dados local e/ou remoto via conexões com ou sem fio;
   \item Segurança para persistência de dados em caso de queda de energia ou pane;
   \item Sincronização de dados com outros sistemas;
   \item Etc.
 \end{itemize}

Atualmente, pensamos em um sistema móvel como um sistema projetado para rodar em um computador de mão, seja ele celular, tablet ou qualquer outro dispositivo com tais características. Pela definição acima, os notebooks também são considerados plataformas para sistemas móveis, mas não são utilizados exatamente da mesma forma que os dispositivos citados acima, pois precisa parar em algum lugar, abrir o notebook, esperar carregar, etc...

\section{Dispositivos móveis}\label{sec:dm}
Para adentrar no universo da solução apresentada, \ac{dm}, é valioso abordar como estes \textit{gadgets} lidam com a capacidade de armazenamento e processamento de dados, que, por possuírem sistemas móveis espera-se que os mesmos possuam menor eficiência para desenvolver atividades se comparados a um computador estacionário (\ac{pc}), por exemplo, que possui muito mais disponibilidade energética para realizar suas tarefas, porém, nos dias de hoje esse \textit{gap} para atividades cotidianas está mais estreito, devido aos grandes avanços da tecnologia voltados a este segmento que vem em forte alta (o qual, segundo \citet{data.ai} cresceu 20\% em 2020 comparado ao ano anterior, gerando um consumo no valor de 143 bilhões de dólares no mundo todo, mesmo com as dificuldades enfrentadas pela pandemia da COVID-19, vale ressaltar que há uma expectativa que o mercado de \acp{dm} cresça ainda mais até 2026 \cite{mordor_intelligence_2021}), já para processamentos mais robustos estamos caminhando em largos passos, atualmente é possível se deparar com sistemas móveis utilizando o processamento em nuvem (que são outra forma de representação de sistema estacionários, um exemplo disso são os servidores computacionais) para tais atividades, enquanto outros \acp{dm} já contam com \textit{chipsets} dedicados exclusivamente para isso.

Para situar onde os \acp{dm} chegaram, é preciso olhar um pouco o passado para lembrar como os computadores eram: máquinas que ocupavam salas gigantes, os quais eram manuseados somente por setores importantes da sociedade, antes mesmo de ser um dispositivo doméstico como é hoje, limitando-se a órgãos do governo, instituições de ensino e poucas empresas, por exemplo. \citet{alecrim_2013}.

Com o desenvolvimento da tecnologia, os computadores tornaram-se cada vez mais compactos, eficientes, práticos e fáceis de usar, podendo ser levados para qualquer lugar, por qualquer pessoa. As tecnologias que fornecem essa maior flexibilidade são conhecidas como \acp{dm}.

Sintetizando, um \ac{dm} é um tipo de dispositivo computacional que tem como principais características a portabilidade, a compactabilidade e fácil manuseio \citet{lee2005aplicaccoes}, além de todos os aspectos citados na seção \ref{sec:mobilidade}. 

De acordo com \citet{laricchia_2022}, em 2021, o número de \acp{dm} operando em todo o mundo ficou em quase 15 bilhões, contra pouco mais de 14 bilhões no ano anterior. Espera-se que o número de \acp{dm} atinja 18,22 bilhões até 2025, um aumento de 4,2 bilhões de dispositivos (aproximadamente 30\%) em comparação com os níveis de 2020. A \figref{fig:mobiles20to25} mostra o gráfico desta previsão.

\begin{figure}[h]
\centering
  \includegraphics[width=\columnwidth]{images/mobiles20to25.png}
  \caption{Previsão do número de dispositivos móveis em todo o mundo de 2020 a 2025 (em bilhões)}
  \label{fig:mobiles20to25}
\end{figure}

Visto esse crescimento no uso de \acp{dm}, é perceptível que a demanda por \textbf{desenvolvimento de \textit{software}} para esta tecnologia também está aumentando. Essas soluções são chamadas de aplicativos móveis, o qual será abordado com mais detalhes na seção \ref{sec:apps} e posteriormente detalhado sobre o mercado de desenvolvimento de aplicativos na subseção \ref{ssec:dev_apps}.

\subsection{iOS}\label{ssec:ios}
\lipsum[1]


\subsection{Android}\label{ssec:android}
\lipsum[1]


\section{Aplicativos móveis}\label{sec:apps}
\lipsum[1]


\subsection{Desenvolvimento de aplicativos}\label{ssec:dev_apps}
\lipsum[1]

\subsubsection{Desenvolvimento nativo}\label{sssec:dev_apps_nativo}
\lipsum[1]

\subsubsection{Desenvolvimento híbrido}\label{sssec:dev_apps_hibrido}
\lipsum[1]


\section{[ abordar sobre o tema do app, area que es'ta associada à agronomia, etc... ]}
\lipsum[1]


\chapter{Metodologia}
Neste capítulo é apresentada a metodologia utilizada para o desenvolvimento do aplicativo. 
[EM DESENVOLVIMENTO até 31/03/23]

% Após o levantamento dos conceitos necessários para o desenvolvimento da aplicação e um estudo de aplicativos similares na literatura, o próximo passo foi um levantamento dos sistemas existente na CBMPA. Revisão bibliográfica: descrever as principais referências utilizadas para a construção da aplicação móvel, incluindo frameworks, bibliotecas e outras tecnologias relacionadas.

% Revisão bibliográfica: descrever as principais referências utilizadas para a construção da aplicação móvel, incluindo frameworks, bibliotecas e outras tecnologias relacionadas.

% Especificação de requisitos: descrever os requisitos funcionais e não funcionais da aplicação móvel, incluindo as funcionalidades que ela deve ter e as plataformas em que deve ser executada.

% Prototipagem: descrever o processo de prototipagem da aplicação móvel, incluindo a criação de wireframes, modelos de tela e fluxos de navegação.

% Desenvolvimento: descrever o processo de desenvolvimento da aplicação móvel, incluindo as ferramentas e tecnologias utilizadas, como a linguagem de programação, banco de dados e plataformas de desenvolvimento.

% Testes: descrever o processo de testes da aplicação móvel, incluindo os tipos de testes realizados, como testes funcionais e de desempenho.

% Implantação: descrever o processo de implantação da aplicação móvel, incluindo a distribuição nas lojas de aplicativos, atualizações e manutenção.


\chapter{Desenvolvimento}

\section{Back-end}

\lipsum[1]

\section{Comunicação por API}

\lipsum[1]

\section{Aplicativo móvel desenvolvido}

% \subsection{Subsection}

\lipsum[1]
\chapter{Conclusão}
[EM DESENVOLVIMENTO até 31/03/23]

\section{Trabalhos futuros}
[EM DESENVOLVIMENTO até 31/03/23]

% \lipsum[1]

% \section{Section}

% \lipsum[1]

% \subsection{Subsection}

% \lipsum[1]

% References

\begin{references}
  \bibliography{references}
\end{references}

% Appendix

% \theappendix
% \include{appendix/mapping-study}

\end{document}
