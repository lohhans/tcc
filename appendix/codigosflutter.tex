\chapter{Códigos Flutter}\label{ch:codigosFlutter}

\section{Código da tela \textit{Home Page} desenvolvida na Terceira Sprint}\label{code:home-sprint3}
\begin{code}
% \captionof{listing}{Home Page}
\begin{minted}{dart}
// imports suprimidos

class HomePage extends StatefulWidget {
  const HomePage({Key? key}) : super(key: key);

  @override
  State<HomePage> createState() => _HomePageState();
}

class _HomePageState extends State<HomePage> {
  late final HomeController controller;
  late final ExperimentsController experimentsController;
  late final TreatmentsController treatmentsController;
  late final AccountController accountController;

  late List<Widget> _fragments;

  @override
  void initState() {
    super.initState();
    controller = context.read<HomeController>();
    experimentsController = context.read<ExperimentsController>();
    treatmentsController = context.read<TreatmentsController>();
    accountController = context.read<AccountController>();
    initFragements();
    if (mounted) {
      Future.delayed(Duration.zero, () async {
        await experimentsController.loadExperiments();
        await treatmentsController.loadTreatments();
        await accountController.loadAccount();
      });
      controller.addListener(() {
        if (controller.state == HomeState.error) {
          ScaffoldMessenger.of(context).showSnackBar(
            SnackBar(
              content: Text(controller.failure!.message),
            ),
          );
        }
      });
    }
  }

  initFragements() {
    _fragments = [
      ExperimentsPage(
        homeController: controller,
      ),
      TreatmentsPage(
        homeController: controller,
      ),
      AccountPage(
        homeController: controller,
      ),
    ];
  }

  Widget? get dealWithFloatingActionButton {
    if (controller.fragmentIndex == 0) {
      return FloatingActionButton.extended(
        onPressed: () {
          Navigator.pushNamed(
            context,
            RouteGenerator.createExperiment,
          );
        },
        label: Text(
          "Cadastrar\nexperimento",
          style: TextStyles.buttonBackground,
        ),
        icon: const Icon(
          PhosphorIcons.pencilLine,
          color: AppColors.white,
          size: 30,
        ),
      );
    }

    if (controller.fragmentIndex == 1) {
      return FloatingActionButton.extended(
        onPressed: () {
          Navigator.pushNamed(
            context,
            RouteGenerator.createTreatment,
          );
        },
        label: Text(
          "Cadastrar\ntratamento",
          style: TextStyles.buttonBackground,
        ),
        icon: const Icon(
          PhosphorIcons.pencilLine,
          color: AppColors.white,
          size: 30,
        ),
      );
    }

    return null;
  }

  @override
  Widget build(BuildContext context) {
    final controller = context.watch<HomeController>();

    return Scaffold(
      appBar: AppBar(
        title: SvgPicture.asset(
          'assets/images/logo.svg',
          fit: BoxFit.contain,
          alignment: Alignment.center,
        ),
      ),
      body: ChangeNotifierProvider(
        create: (BuildContext context) {},
        child: _fragments[controller.fragmentIndex],
      ),
      floatingActionButton: dealWithFloatingActionButton,
      bottomNavigationBar: BottomNavigationBar(
        items: const [
          BottomNavigationBarItem(
            icon: Icon(PhosphorIcons.flask),
            label: 'Experimentos',
          ),
          BottomNavigationBarItem(
            icon: Icon(PhosphorIcons.testTube),
            label: 'Tratamentos',
          ),
          BottomNavigationBarItem(
            icon: Icon(PhosphorIcons.userCircleGear),
            label: 'Conta',
          ),
        ],
        currentIndex: controller.fragmentIndex,
        selectedItemColor: AppColors.white,
        backgroundColor: AppColors.primary,
        onTap: (index) => controller.setFragmentIndex(index),
      ),
    );
  }
}
\end{minted}
\end{code}

\section{Trecho de código para realizar a checagem de autenticação e pré-carregamento dos dados desenvolvido na Sprint 6}\label{code:check-auth}
\begin{code}
% \captionof{listing}{Trecho de código para realizar a checagem de autenticação e pré-carregamento dos dados}
% \label{code:dart-code}
\begin{minted}{dart}
_checkAuth() async {
    Future.delayed(Duration.zero).then((_) async {
    String token = await getToken() ?? '';

    if (!mounted) return;

    if (token.isEmpty) {
        Navigator.pushReplacementNamed(context, RouteGenerator.auth);
    } else {
        await Provider.of<HomeViewmodel>(context, listen: false)
        .getContent()
        .then((value) =>
            Navigator.pushReplacementNamed(
                context, 
                RouteGenerator.home,
            ),
        );
      }
    });
  }
\end{minted}
\end{code}