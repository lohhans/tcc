Diante da projeção do crescimento da população mundial, a \ac{qs} é um dos fatores fundamentais para o aumento da produtividade agrícola e para a redução da escassez global de alimentos. Nesse contexto, é fundamental entender a \ac{qs}, suas características físicas, químicas e biológicas, bem como as interações entre esses fatores, permitindo o desenvolvimento de técnicas mais eficientes e sustentáveis de manejo do solo. Diversos métodos são utilizados para avaliar a \ac{qs}, entre eles, o cálculo de \acp{ae} do solo é uma importante ferramenta para avaliar a qualidade e saúde do solo, uma vez que a \acp{ae} está diretamente relacionada com a quantidade e diversidade de microrganismos presentes no solo. Desta forma, é de fundamental importância a disponibilidade de informação precisa e também de sistemas computacionais para cálculo das \acp{ae} e para auxílio na tomada de decisão. Dessa maneira, este trabalho propõe um sistema para o gerenciamento de experimentos e resultados das análises das \acp{ae} do solo. Esta abordagem foi implementada através de conceitos como métodos ágeis e com arquitetura limpa, contemplando inicialmente o cálculo das enzimas fosfatase ácida e alcalina, a $\beta$-glucosidase, a urease e a protease. De forma que, após a análise da viabilidade e especificação dos requisitos do produto, junto aos envolvidos no desenvolvimento deste software, foi obtido um protótipo, do qual a solução foi concebida. Sendo assim, esta ferramenta irá auxiliar pesquisadores, laboratórios e profissionais da área, para que possam realizar experimentos de solo com embasamento científico de forma prática e precisa, aperfeiçoando as práticas de manejo.

\begin{keywords}
Qualidade do solo, Enzimas extracelulares, Engenharia de Software, Desenvolvimento Mobile, Metodologias Ágeis, Arquitetura Limpa, Flutter.
\end{keywords}