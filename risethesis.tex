%% RiSE Latex Template - version 0.5
%%
%% RiSE's latex template for thesis and dissertations
%% http://risetemplate.sourceforge.net
%%
%% (c) 2012 Yguaratã Cerqueira Cavalcanti (yguarata@gmail.com)
%%          Vinicius Cardoso Garcia (vinicius.garcia@gmail.com)
%% 
%% (!) Recent modifications to UFAPE by: 
%%     Armstrong Lohãns de Melo Gomes Quintino (lohansdemelo1108@gmail.com)
%%
%% This document was initially based on UFPEThesis template, from Paulo Gustavo
%% S. Fonseca.
%%
%% ACKNOWLEDGEMENTS
%%
%% We would like to thanks the RiSE's researchers community, the 
%% students from Federal University of Pernambuco, and other users that have
%% been contributing to this projects with comments and patches.
%%
%% GENERAL INSTRUCTIONS
%%
%% We strongly recommend you to compile your documents using pdflatex command.
%% It is also recommend use the texlipse plugin for Eclipse to edit your documents.
%%
%% Options for \documentclass command:
%%         * Idiom
%%           pt   - Portguese (default)
%%           en   - English
%%
%%         * Text type
%%           bsc  - B.Sc. Thesis
%%           msc  - M.Sc. Thesis (default)
%%           qual - PHD qualification (not tested yet)
%%           prop - PHD proposal (not tested yet)
%%           phd  - PHD thesis
%%
%%         * Media
%%           scr  - to eletronic version (PDF) / see the users guide
%%
%%         * Pagination
%%           oneside - unique face press
%%           twoside - two faces press
%%
%%		   * Line spacing
%%           singlespacing  - the same as using \linespread{1}
%%           onehalfspacing - the same as using \linespread{1.3}
%%           doublespacing  - the same as using \linespread{1.6}
%%
%% Reference commands. Use the following commands to make references in your
%% text:
%%          \figref  -- for Figure reference
%%          \tabref  -- for Table reference
%%          \eqnref  -- for equation reference
%%          \chapref -- for chapter reference
%%          \secref  -- for section reference
%%          \appref  -- for appendix reference
%%          \axiref  -- for axiom reference
%%          \conjref -- for conjecture reference
%%          \defref  -- for definition reference
%%          \lemref  -- for lemma reference
%%          \theoref -- for theorem reference
%%          \corref  -- for corollary reference
%%          \propref -- for proprosition reference
%%          \pgref   -- for page reference
%%
%%          Example: See \chapref{chap:introduction}. It will produce 
%%                   'See Chapter 1', in case of English language.

\documentclass[pt,oneside,onehalfspacing,bsc]{risethesis}

% \usepackage[english]{babel}
\usepackage[portuguese]{babel}
\usepackage{colortbl}
\usepackage{color}
\usepackage[table]{xcolor}
\usepackage{microtype}
\usepackage{bibentry}
\usepackage{subfigure}
\usepackage{multirow}
\usepackage{rotating}
\usepackage{booktabs}
\usepackage{pdfpages}
\usepackage{caption}
\usepackage{lipsum}
\usepackage{url}
\usepackage{natbib}
% Packages adicionados
\usepackage{nicefrac} % For comparison
\usepackage{xfrac}    % Works better with other fonts
\usepackage{float}
\usepackage{graphicx} % Tabela
\usepackage{tabularx}
\usepackage{tcolorbox}
\usepackage[newfloat]{minted} % Codigo
\usepackage{subcaption}
\usepackage{tabularray}
\usepackage{longtable}
\usepackage{geometry}
\usepackage[utf8]{inputenc}
\usepackage{makecell}

\newenvironment{code}{\captionsetup{type=listing}}{}
\SetupFloatingEnvironment{listing}{name=Código Fonte}

\newtcbox{\inlinecode}{on line, boxrule=0pt, boxsep=0pt, top=2pt, left=2pt, bottom=2pt, right=2pt, colback=gray!15, colframe=white, fontupper={\ttfamily \footnotesize}}

\captionsetup[table]{position=top,justification=centering,width=.85\textwidth,labelfont=bf,font=small}
\captionsetup[lstlisting]{position=top,justification=centering,width=.85\textwidth,labelfont=bf,font=small}
\captionsetup[figure]{position=bottom,justification=centering,width=.85\textwidth,labelfont=bf,font=small}

%% Change the following pdf author attribute name to your name.
\usepackage[linkcolor=black,
            citecolor=blue,
            urlcolor=black,
            colorlinks,
            pdfpagelabels,
            pdftitle={Sistema de experimentos para análise e cálculo de atividades enzimáticas do solo},
            pdfauthor={Armstrong Lohãns de Melo Gomes Quintino}]{hyperref}

% \input{./listings_dart}

\address{Garanhuns-PE}

\universitypt{Universidade Federal do Agreste de Pernambuco}
\universityen{Federal University of the Agreste of Pernambuco}

\departmentpt{}
\departmenten{}
% \departmentpt{Centro de Informática}
% \departmenten{Center for Informatics}

\programpt{Graduação em Ciência da Computação}
\programen{Graduate in Computer Science}

\majorfieldpt{Ciência da Computação}
\majorfielden{Computer Science}

%\title{Enzitech: Sistema de experimentação para análise e cálculo de atividades enzimáticas do solo}

\title{Sistema de gestão de experimentos e cálculo de atividades enzimáticas do solo}

%\title{Sistema para Análise e Cálculo de atividades enzimáticas do solo}


\date{2023}

\author{Armstrong Lohãns de Melo Gomes Quintino}
\adviser{Rodrigo Gusmão de Carvalho Rocha}

% Macros (defines your own macros here, if needed)
\def\x{\checkmark}

\begin{document}

\frontmatter

\frontpage

\presentationpage

\begin{fichacatalografica}
	\FakeFichaCatalografica % Comment this line when you have the correct file
%     \includepdf{fig_ficha_catalografica.pdf} % Uncomment this
\end{fichacatalografica}

\banca
% \begin
  % Jean Carlos de Araújo Teixeira
  % Ícaro Lins Leitão da Cunha
  % Jamilly Barros


\begin{dedicatory}
Eu dedico este trabalho a todos os meus familiares, amigos e colegas que sempre acreditaram em mim, dedico em especial ao meu avô paterno, Vovô Zezinho, que não está mais aqui, e minha avó materna, Vó Augusta, a quem tenho grande apreço e orgulho de poder compartilhar esta conquista.
\end{dedicatory}

\acknowledgements
% TODO: Mudar
Primeiramente a Deus, por ter me concedido o conhecimento, a saúde e a força de vontade necessários para alcançar até aqui.

Aos meus pais, Ricarte Gomes Quintino e Ivanize Alves de Melo Quintino, que batalharam duro, nunca mediram esforços para minha educação e sempre me ensinaram a lutar para atingir meus objetivos, meus primeiros amigos, que me ensinaram valores que levarei para toda a vida. Essa conquista é nossa, amo vocês!

A minha irmã, Roane Lohaynne, a qual sempre foi um espelho como irmã mais velha, que sempre esteve ao meu lado. Você é uma fonte de inspiração e admiração para mim, e tenho muito orgulho de ser seu irmão. Dedico este trabalho a você com todo o meu amor e gratidão.

A minha namorada, Ana Beatriz, que me apoiou em todas as fases dessa jornada. Agradeço por ter estado ao meu lado em todos os momentos, sempre acreditando e me incentivando a buscar o meu melhor. Pude iniciar e concluir esta graduação com você ao meu lado, e espero que possamos continuar juntos nos próximos ciclos da vida. Eu te amo muito, você é um exemplo de amor e companheirismo!

Aos meus queridos amigos do OBDGUS e amigos agregados, em especial Alessandro, Andrei, Cézar, Elyson, Erik e Morgan, por estarem juntos comigo sempre que precisei. Todos vocês participaram de algo na minha graduação — me apoiaram com todas as dificuldades que encontrei no caminho e nunca desacreditaram — mas acima de tudo, participaram da minha vida, como grandes amigos que levarei comigo para sempre. Que venham mais peladas, acampamentos, brejas e partidas de CS! \textit{Tamo junto}!

Aos meus amigos de turma e do nosso grupo "O Núcleo", formado lá no começo do curso, em especial meus amigos Anderson, André, Antônio, Baltazar, Daniel, Igor, Júnior (CJ), Laisy, Luis, Machado, Saú, Valdir e Victor, que contribuíram com minha formação, e apesar de seguirmos um pouco distantes, sei que posso contar com muitos, obrigado pelos anos de conversas e conhecimentos trocados. Carrego todos vocês no peito.

Aos demais amigos e colegas que fiz na graduação, aqui em especial Matheus Noronha e Weverton Cintra, vocês me ajudaram muito nessa fase final do curso, devo muito a vocês, agradeço demais pelas horas disponibilizadas em meio a tanta coisa, sem a ajuda de vocês isso aqui não seria possível.

A todos os professores que me fizeram ser um ser humano melhor, desde o ensino infantil até aqui, principalmente aos professores que fazem parte da nossa universidade e que me instruíram de forma excelente para os desafios da vida profissional.

E claro, agradeço ao professor orientador, Dr. Rodrigo Gusmão de Carvalho Rocha por todo o apoio ao longo do curso, em especial, nesta reta final, pegando na minha mão e me ajudando a alcançar este feito, me auxiliando com todos os contratempos enfrentados.

Para finalizar, a todos os meus colegas e amigos, que não mencionei aqui, mas que de alguma forma participaram deste ciclo maravilhoso. Obrigado!


\begin{epigraph}[]{ Mewtwo}
``As circunstâncias do nascimento de alguém são irrelevantes; é o que você faz com seu dom da vida que determina quem você é.``
\end{epigraph}

\resumo
% Escreva seu resumo no arquivo resumo.tex
\lipsum[1]

\begin{keywords}
Cálculo de atividades enzimáticas do solo, Engenharia de Software, Desenvolvimento Mobile, Arquitetura Limpa, Flutter
\end{keywords}

\abstract
% Write your abstract in a file called abstract.tex
\lipsum[1]

\begin{keywords}
Calculation of soil enzymatic activities, Software Engineering, Mobile Development, Clean Architecture, Flutter
\end{keywords}

% List of figures
\listoffigures

% List of tables
\listoftables

% List of acronyms
% Acronyms manual: http://linorg.usp.br/CTAN/macros/latex/contrib/acronym/acronym.pdf
\listofacronyms
\begin{acronym}[ACRONYM] 
% Change the word ACRONYM above to change the acronym column width.
% The column width is equals to the width of the word that you put.
% Read the manual about acronym package for more examples:
%   http://linorg.usp.br/CTAN/macros/latex/contrib/acronym/acronym.pdf
\acro{ae}[AE]{Atividade Enzimática}
\acrodefplural{ae}{Atividades Enzimáticas}
\acro{am}[AM]{Aprendizado de Máquina}
\acro{app}[APP]{Aplicativo}
\acro{api}[API]{\textit{Application Programming Interface}}
% \acro{apm}[APM]{Aplicativo Móvel}
% \acrodefplural{apm}{Aplicativos Móveis}
\acro{ba}[BA]{Bacharelado em Agronomia}
\acro{bcc}[BCC]{Bacharelado em Ciência da Computação}
\acro{cagr}[CAGR]{Taxa de Crescimento Anual Composta}
\acro{crud}[CRUD]{\textit{Create, Read, Update and Delete}}
\acro{dto}[DTO]{\textit{Data Transfer Object}}
\acro{dm}[DM]{Dispositivo Móvel}
\acrodefplural{dm}{Dispositivos Móveis}
\acro{fao}[FAO]{Organização das Nações Unidas para Agricultura e Alimentação}
\acro{http}[HTTP]{\textit{Hypertext Transfer Protocol}}
\acro{ia}[IA]{Inteligência Artificial}
\acro{ide}[IDE]{\textit{Integrated Development Environment}}
\acro{json}[JSON]{\textit{JavaScript Object Notation}}
\acro{mvp}[MVP]{Produto Viável Mínimo}
\acro{mvvm}[MVVM]{\textit{Model–view–viewmodel}}
\acro{ndc}[NDC]{Contribuição Nacionalmente Determinada}
\acro{onu}[ONU]{Organização das Nações Unidada}
\acro{pc}[PC]{Computador Pessoal}
\acrodefplural{pc}{Computadores Pessoais}
\acro{planoabc}[Plano ABC]{Plano de Agricultura de Baixa Emissão de Carbono}
\acro{rest}[REST]{\textit{Representational State Transfer}}
\acro{po}[PO]{\textit{Product Owner}}
\acro{qs}[QS]{Qualidade do Solo}
\acro{sdk}[SDK]{\textit{Software Development Kit}}
\acro{so}[SO]{Sistema Operacional}
\acrodefplural{so}{Sistemas Operacionais}
\acro{tcc}[TCC]{Trabalho de Conclusão de Curso}
\acro{ufape}[UFAPE]{Universidade Federal do Agreste de Pernambuco}
\acro{ui}[UI]{\textit{User Interface}}
\acro{uml}[UML]{\textit{Unified Modeling Language}}
\acro{url}[URL]{\textit{Uniform Resource Locator}}
\acro{xml}[XML]{\textit{Extensible Markup Language}}
% \acro{soho}[SOHO]{Small Home Office}
% \acro{api}[API]{Application Programming Interface}
% \acro{arima}[ARIMA]{Auto-Regressive Integrated Moving Average}
% \acro{brn}[BRN]{Bug Report Network}
% \acro{bts}[BTS]{Bug Triage System}
% \acro{cas}[CAS]{Context-Aware Systems}
% \acro{ccb}[CCB]{Change Control Board}
\acro{cr}[CR]{Change Request}
% \acro{cvs}[CVS]{Concurrent Version System}
% \acro{es}[ES]{Expert System}
% \acro{floss}[FLOSS]{Free/Libre Open Source Software}
% \acro{glr}[GLR]{Generalized Linear Regression}
% \acro{gqm}[GQM]{Goal Question Metric}
% \acro{html}[HTML]{HyperText Markup Language}
% \acro{ir}[IR]{Information Retrieval}
% \acro{irt}[IRT]{Recôncavo Institute of Technology}
% \acro{jdt}[JDT]{Jazz Duplicate Finder}
% \acro{lda}[LDA]{Latent Dirichlet Allocation}
% \acro{loc}[LOC]{Lines of Code}
% \acro{lsi}[LSI]{Latent Semantic Indexing}
% \acro{ms}[MS]{Mapping Study}
% \acro{msr}[MSR]{Mining Software Repositories}
% \acro{nlp}[NLP]{Natural Language Processing}
% \acro{promise}[PROMISE]{Predictive Models in Software Engineering}
% \acro{rbes}[RBES]{Rule-Based Expert System}
% \acro{rhel}[RHEL]{RedHat Enterprise Linux}
% \acro{saas}[SaaS]{Software as a Service}
% \acro{scm}[SCM]{Software Configuration Management}
% \acro{serpro}[SERPRO]{Brazilian Federal Organization for Data Processing}
% \acro{slr}[SLR]{Stepwise Linear Regression}
% \acro{slr}[SLR]{Systematic Literature Review}
% \acro{svd}[SVD]{Singular Value Decomposition}
% \acro{svm}[SVM]{Support Vector Machine}
% \acro{svn}[SVN]{Subversion}
% \acro{tfidf}[TF-IDF]{Term Frequency-Inverse Document Frequency}
% \acro{vsm}[VSM]{Vector Space Model}
% \acro{xp}[XP]{Extreming Programming}
\end{acronym}

% Summary (tables of contents)
\tableofcontents

\mainmatter

\chapter{Introdução}
\label{chp:introduction}
A eficiente e incessante evolução da tecnologia traz consigo a grande oportunidade — como também a necessidade — de tornar as tarefas cotidianas dos seres humanos cada vez mais triviais e informatizadas, ou seja, realizar uma tarefa de forma que exija o menor esforço possível, dessa forma, algo que requisitaria tempo e atenção de alguém vem a se tornar uma tarefa simples para um computador.

As aplicações móveis, habitualmente conhecidas por \acp{app}, são exemplos claros e amplamente difundidos atualmente, visto que cada vez mais tarefas, desde o envio de mensagens até o pagamento de contas, podem ser realizadas nas palmas das nossas mãos, basicamente utilizando um \textit{smartphone} com acesso à Internet. O que há pouco mais de uma década era visualizado como uma atividade exclusiva tecnologicamente de um \ac{pc}, fosse através de uma aplicação \textit{web} ou \textit{desktop}, hoje é facilmente feito por soluções adaptadas ou similares para as telas dos celulares, tornando-se uma atividade corriqueira na vida da maioria das pessoas.

Em outra esfera, a análise de \acp{ae} do solo é um importante indicador da qualidade e saúde do solo, pois essas atividades estão diretamente relacionadas com a disponibilidade de nutrientes e a capacidade do solo em sustentar a vida vegetal. Porém, as técnicas convencionais para a medição dessas atividades são trabalhosas, caras e requerem equipamentos sofisticados.

Segundo \cite{nascimento2017impacto}, o uso de tecnologias nas instituições de pesquisa científica em agronomia tem sido amplamente adotado nas últimas décadas, com o objetivo de aumentar a eficiência e a precisão dos experimentos, bem como de facilitar a coleta e análise de dados. A implementação de sistemas informatizados e a utilização de equipamentos avançados de análise, como espectrômetros e microscópios eletrônicos, são exemplos dessas tecnologias que vêm sendo empregadas em estudos agronômicos.

Nesse contexto, o desenvolvimento de um aplicativo móvel para a realização de experimentos de análise e cálculo de \acp{ae} do solo pode ser uma solução eficaz e prática para essa demanda. O objetivo deste trabalho é apresentar o desenvolvimento de um aplicativo móvel que permite a realização destes experimentos, com o intuito de facilitar e tornar mais acessível a análise dessas atividades, buscando harmonizar o crescimento econômico, a equidade social e a preservação ambiental, visando à conservação do meio ambiente e à garantia do bem-estar humano a longo prazo.

\section[Motivação]{Motivação}
Os experimentos do solo com cálculo de \acp{ae} são uma técnica utilizada em estudos de ciência do solo para avaliar a saúde e qualidade do solo, estes experimentos apresentam alguns desafios e lacunas, como a dificuldade na padronização dos métodos de análise, a falta de informações sobre a relação entre as \acp{ae} e a fertilidade do solo e a dificuldade em avaliar a dinâmica temporal das \acp{ae}. Além disso, há pouco conhecimento sobre as \acp{ae} em diferentes tipos de solos e em diferentes regiões do mundo, e a identificação de microrganismos e enzimas específicas pode ser desafiadora. Esses desafios podem afetar a interpretação dos resultados dos experimentos e limitar a aplicação das técnicas em diferentes contextos \cite{tabatabai1994soil}.

Este tema é importante para a comunidade acadêmica e para a sociedade em geral, pois permite avaliar a saúde e qualidade do solo, identificar microrganismos e enzimas benéficos para o solo e para as plantas, promovendo práticas agrícolas mais sustentáveis e eficientes, além de monitorar a qualidade do solo em áreas urbanas e industriais. A pesquisa e desenvolvimento do sistema proposto para essa área contribui para o desenvolvimento sustentável e para a conservação do meio ambiente.
 
\section{Objetivo Geral}\label{sec:objetivo_geral}

Este trabalho tem como objetivo detalhar a solução e o processo de desenvolvimento de uma aplicação para dispositivos móveis que informatize o método de fazer experimentos e análises do solo, aplicativo este desenvolvido usando padrões arquiteturais limpos, escaláveis, manuteníveis e testáveis.

\section{Objetivos específicos}\label{sec:objetivo_específico}
Com base no objetivo geral, correspondem os objetivos específicos indicados a seguir:

\begin{itemize}
    \item Implementar um sistema, acessível via aplicativo móvel, que permite criar e gerenciar experimentos que realizem cálculos das \acp{ae} do solo;
    \item Gerar resultados e planilhas de forma automática;
    \item Possibilitar a análise dos resultados;
    \item Proporcionar uma experiência fluida de navegação no aplicativo, ao utilizar boas práticas de desenvolvimento de \textit{software}.
\end{itemize}

\section{Organização do texto}

O presente trabalho foi assim constituído:
\begin{itemize}
    \item O Capítulo \ref{ch:teoria} discorre sobre conceitos básicos dos principais fundamentos adotados neste trabalho e relacionados ao tema científico, em que são abordado os temas de \acp{ae} do solo, dispositivos móveis, desenvolvimento de aplicativos móveis e arquitetura limpa;
    \item No Capítulo \ref{ch:metodos} é realizada uma apresentação dos métodos utilizados no desenvolvimento da solução, bem como os requisitos do sistema, tecnologias e ferramentas envolvidas;
    \item No Capítulo \ref{ch:proposta}, são apresentados os processos do desenvolvimento, arquiteturas utilizadas, informações sobre os testes e implantação do aplicativo e do sistema em si;
    \item No Capítulo \ref{ch:resultados}, é apresentado a versão final do aplicativo desenvolvido, o fluxo de navegação e como todo o sistema se comporta;
    \item No Capítulo \ref{ch:conclusao}, são apresentadas as considerações finais, principais contribuições, dificuldades encontradas e trabalhos futuros.
\end{itemize}

\chapter{Fundamentação Teórica}

\section{Mobilidade}\label{sec:mobilidade}
Segundo \citet{b2004mobile}, um sistema de computação móvel é um sistema que pode ser facilmente movido fisicamente ou cuja funcionalidade pode ser usada durante o movimento. Como esses sistemas fornecem essa mobilidade, essas funcionalidades adicionadas são a razão para caracterizar separadamente os sistemas de computação móvel, eles geralmente oferecem capacidades e recursos não encontrados em sistemas normais, como: 
 \begin{itemize}
   \item Armazenamento de dados local e/ou remoto via conexões com ou sem fio;
   \item Segurança para persistência de dados em caso de queda de energia ou pane;
   \item Sincronização de dados com outros sistemas;
   \item Etc.
 \end{itemize}

Atualmente, pensamos em um sistema móvel como um sistema projetado para rodar em um computador de mão, seja ele celular, tablet ou qualquer outro dispositivo com tais características. Pela definição acima, os notebooks também são considerados plataformas para sistemas móveis, mas não são utilizados exatamente da mesma forma que os dispositivos citados acima, pois precisa parar em algum lugar, abrir o notebook, esperar carregar, etc...

\section{Dispositivos móveis}\label{sec:dm}
Para adentrar no universo da solução apresentada, \ac{dm}, é valioso abordar como estes \textit{gadgets} lidam com a capacidade de armazenamento e processamento de dados, que, por possuírem sistemas móveis espera-se que os mesmos possuam menor eficiência para desenvolver atividades se comparados a um computador estacionário (\ac{pc}), por exemplo, que possui muito mais disponibilidade energética para realizar suas tarefas, porém, nos dias de hoje esse \textit{gap} para atividades cotidianas está mais estreito, devido aos grandes avanços da tecnologia voltados a este segmento que vem em forte alta (o qual, segundo \citet{data.ai} cresceu 20\% em 2020 comparado ao ano anterior, gerando um consumo no valor de 143 bilhões de dólares no mundo todo, mesmo com as dificuldades enfrentadas pela pandemia da COVID-19, vale ressaltar que há uma expectativa que o mercado de \acp{dm} cresça ainda mais até 2026 \cite{mordor_intelligence_2021}), já para processamentos mais robustos estamos caminhando em largos passos, atualmente é possível se deparar com sistemas móveis utilizando o processamento em nuvem (que são outra forma de representação de sistema estacionários, um exemplo disso são os servidores computacionais) para tais atividades, enquanto outros \acp{dm} já contam com \textit{chipsets} dedicados exclusivamente para isso.

Para situar onde os \acp{dm} chegaram, é preciso olhar um pouco o passado para lembrar como os computadores eram: máquinas que ocupavam salas gigantes, os quais eram manuseados somente por setores importantes da sociedade, antes mesmo de ser um dispositivo doméstico como é hoje, limitando-se a órgãos do governo, instituições de ensino e poucas empresas, por exemplo. \citet{alecrim_2013}.

Com o desenvolvimento da tecnologia, os computadores tornaram-se cada vez mais compactos, eficientes, práticos e fáceis de usar, podendo ser levados para qualquer lugar, por qualquer pessoa. As tecnologias que fornecem essa maior flexibilidade são conhecidas como \acp{dm}.

Sintetizando, um \ac{dm} é um tipo de dispositivo computacional que tem como principais características a portabilidade, a compactabilidade e fácil manuseio \citet{lee2005aplicaccoes}, além de todos os aspectos citados na seção \ref{sec:mobilidade}. 

De acordo com \citet{laricchia_2022}, em 2021, o número de \acp{dm} operando em todo o mundo ficou em quase 15 bilhões, contra pouco mais de 14 bilhões no ano anterior. Espera-se que o número de \acp{dm} atinja 18,22 bilhões até 2025, um aumento de 4,2 bilhões de dispositivos (aproximadamente 30\%) em comparação com os níveis de 2020. A \figref{fig:mobiles20to25} mostra o gráfico desta previsão.

\begin{figure}[h]
\centering
  \includegraphics[width=\columnwidth]{images/mobiles20to25.png}
  \caption{Previsão do número de dispositivos móveis em todo o mundo de 2020 a 2025 (em bilhões)}
  \label{fig:mobiles20to25}
\end{figure}

Visto esse crescimento no uso de \acp{dm}, é perceptível que a demanda por \textbf{desenvolvimento de \textit{software}} para esta tecnologia também está aumentando. Essas soluções são chamadas de aplicativos móveis, o qual será abordado com mais detalhes na seção \ref{sec:apps} e posteriormente detalhado sobre o mercado de desenvolvimento de aplicativos na subseção \ref{ssec:dev_apps}.

\subsection{iOS}\label{ssec:ios}
\lipsum[1]


\subsection{Android}\label{ssec:android}
\lipsum[1]


\section{Aplicativos móveis}\label{sec:apps}
\lipsum[1]


\subsection{Desenvolvimento de aplicativos}\label{ssec:dev_apps}
\lipsum[1]

\subsubsection{Desenvolvimento nativo}\label{sssec:dev_apps_nativo}
\lipsum[1]

\subsubsection{Desenvolvimento híbrido}\label{sssec:dev_apps_hibrido}
\lipsum[1]


\section{[ abordar sobre o tema do app, area que es'ta associada à agronomia, etc... ]}
\lipsum[1]


\chapter{Métodos Utilizados}\label{ch:metodos}
Este capítulo descreve a metodologia aplicada no desenvolvimento do trabalho, que tem como objetivo uma solução para dispositivos móveis para a realização de experimentos das atividades enzimáticas do solo. Para realizar o trabalho foram utilizados os conceitos de engenharia de requisitos como parte inicial do projeto, a modelagem do sistema por meio de requisitos funcionais e casos de uso, e a definição dos processos de desenvolvimento do software que serão apresentados nas próximas seções.

\section{Análise de Viabilidade}\label{sec:analise_viabilidade}
Nesta fase foi feita uma estimativa das diversas maneiras de suprir as necessidades do cliente, identificando tanto a parte tecnológica como a parte de técnica. Foi verificado o retorno acadêmico da ferramenta, os avanços e melhorias no processo de experimentação do solo que seriam possíveis obter com a mesma.

\section{Laboratório BCC Coworking}\label{sec:lab}
O projeto teve início na disciplina de Desenvolvimento Distribuído de Software, no curso de \ac{bcc} da \ac{ufape}, o qual, logo em seguida, teve seu escopo de desenvolvimento integrado com o Laboratório \ac{bcc} Coworking, da mesma instituição, este que é um Laboratório de Pesquisa e Desenvolvimento projetado por docentes deste curso, o propósito deste espaço é fornecer um ambiente apropriado para o desenvolvimento de projetos reais, com a supervisão de profissionais da área para garantir o aprimoramento do conhecimento e experiência técnica. Este espaço é destinado aos estudantes que buscam autonomia, estímulo e desenvolvimento de projetos, visando melhorar sua produtividade e expandir seu conhecimento prático.

\section{Equipe de desenvolvimento do Enzitech}
Segundo \cite{pressman2016engenharia}, a análise de mercado, a análise competitiva e a análise SWOT (que envolve a avaliação das forças, fraquezas, oportunidades e ameaças da empresa) são métodos comuns para identificar e avaliar oportunidades de produto na área de tecnologia. Essas análises são importantes para avaliar as necessidades do mercado, os produtos concorrentes, as forças e fraquezas da empresa e as oportunidades e ameaças. 

A ideia do Enzitech foi concebida através de uma observação de melhoria dentre os docentes e discentes de agronomia, ao realizarem as atividades de campo, onde dados eram coletados em papéis e depois passados para uma planilha para o cálculo e coleta de resultados, assim, o Laboratório de Pesquisa em Solo da \ac{ufape} identificou a necessidade de desenvolver uma solução que pudesse aprimorar seus processos, portanto, recorreu à expertise do Laboratório BCC Coworking, reconhecido por suas soluções inovadoras e de alta qualidade, para que juntos pudessem desenvolver uma solução que atendesse a essas demandas específicas. 

A partir daí o Enzitech já estava sendo moldado, onde a equipe foi formada por Armstrong Lohãns, Matheus Noronha, Weverton Cintra, José Vieira, Eduarda Interaminense, e as professoras Erika Valente e Jamille Barros — do curso de Agronomia — que respectivamente propuseram o sistema. A equipe do Enzitech realizou várias reuniões para obter um conjunto inicial de requisitos e elaborar uma breve apresentação da ideia do sistema, além de discutir como o sistema poderia melhorar os processos na área da Agronomia.

\section{Especificação de requisitos}\label{sec:especificacao_de_requisitos}
A engenharia de requisitos é um processo fundamental na concepção e construção de sistemas, pois busca definir os requisitos do sistema em conjunto com os \textit{stakeholders}, produzindo uma série de artefatos em suas fases de concepção. Segundo \cite{kotonya1998requirements}, a engenharia de requisitos é o processo de descobrir, analisar, documentar e verificar serviços e restrições para um sistema. Embora não haja uma maneira irrefutável de garantir que as especificações construídas pela engenharia de requisitos sejam totalmente compatíveis com as necessidades dos \textit{stakeholders} e atendam às suas necessidades, este é o maior desafio encontrado na engenharia de requisitos \cite{pressman2016engenharia}. As etapas da engenharia de requisitos incluem a concepção e análise de requisitos, levantamento e compreensão dos requisitos junto ao cliente, negociação dos requisitos, especificação e modelagem desses requisitos, tendo como objetivo documentar as necessidades e os propósitos do software.

O pontapé inicial do projeto em si foi uma reunião, como proposto nos \textit{frameworks} ágeis, entre as partes envolvidas, cliente (docentes de Agronomia da \ac{ufape}), \ac{po} (Proprietário do produto, em tradução livre) e desenvolvedores, a fim de levantar os requisitos funcionais do \ac{mvp}. Nesta reunião foram explanados os pontos onde o \ac{app} auxiliaria no gerenciamento de experimentos e no processo dos cálculos enzimáticos, foram discutidas regras de negócios e requisitos necessários, identidade visual, além do início de um protótipo de baixa fidelidade para conduzir o \textit{brainstorming} a respeito de como os requisitos se transformariam em funcionalidades, e claro, as tecnologias utilizadas em todo o ciclo de desenvolvimento.

O projeto surge com a necessidade de um \ac{app} para celulares, facilitando a mobilidade dos usuários nos locais de coletas de dados para os experimentos, dessa forma, tudo foi pensado voltado ao desenvolvimento deste aplicativo móvel.

\subsection{Requisitos funcionais}\label{ssec:requisitos_funcionais}
A partir disso, os requisitos funcionais do sistema foram definidos e especificados de acordo com a \tabref{table:requisitos_funcionais}, abaixo.

\begin{table}[H]
\centering
\resizebox{\textwidth}{!}{%
\begin{tabular}{|l|l|}
\hline

% % \DefTblrTemplate{caption}{default}{Teste 1}    % Removes a caption
% \DefTblrTemplate{capcont}{default}{Continuação da página anterior}    % Removes a caption on subsequent pages
% \DefTblrTemplate{contfoot}{default}{Continua na próxima página}   % Removes text denoting continuation on next

% \begin{longtblr}[
%   caption = {Descrição dos requisitos funcionais do sistema}
%   label = {table:requisitos_funcionais},
% ]{
%   colspec = {|XX[4]|},
%   rowhead = 1,
%   hlines,
%   row{even} = {gray9},
%   row{1} = {olive9},
% } 
\textbf{Código} & \textbf{Descrição} \\ \hline
\textbf{RF01}   & O usuário deve ser capaz de se cadastrar no sistema \\ \hline
\textbf{RF02}   & O usuário deve ser capaz de realizar \textit{login} no sistema \\ \hline
\textbf{RF03}   & O usuário deve ser capaz de recuperar sua senha no sistema \\ \hline
\textbf{RF04}   & O usuário deve ser capaz de criar um tratamento \\ \hline
\textbf{RF05}   & O usuário deve ser capaz de deletar um tratamento \\ \hline
\textbf{RF06}   & \vtop{\hbox{\strut O usuário deve ser capaz de visualizar todos os tratamentos disponíveis}\hbox{\strut no sistema para seus experimentos}} \\ \hline
\textbf{RF07}   & O usuário administrador deve ser capaz de criar uma nova enzima \\ \hline
\textbf{RF08}   & O usuário administrador deve ser capaz de deletar uma enzima \\ \hline
\textbf{RF09}   & \vtop{\hbox{\strut O usuário deve ser capaz de visualizar todas as enzimas disponíveis}\hbox{\strut no sistema para seus experimentos}} \\ \hline
\textbf{RF10}   & O usuário deve ser capaz de criar um novo experimento \\ \hline
\textbf{RF11}   & O usuário deve ser capaz de deletar um experimento \\ \hline
\textbf{RF12}   & O usuário deve ser capaz de visualizar seus experimentos cadastrados \\ \hline
\textbf{RF13}   & \vtop{\hbox{\strut O usuário deve ser capaz de utilizar filtros de busca}\hbox{\strut para visualizar seus experimentos cadastrados}} \\ \hline
\textbf{RF14}   & O usuário deve ser capaz de visualizar detalhes dos seus experimentos cadastrados \\ \hline
\textbf{RF15}   & \vtop{\hbox{\strut O usuário deve ser capaz de inserir dados para o cálculo enzimático}\hbox{\strut no seu experimento}} \\ \hline
\textbf{RF16}   & \vtop{\hbox{\strut O usuário deve ser capaz de visualizar resultados discrepantes do cálculo}\hbox{\strut enzimático antes de salvar no seu experimento}}  \\ \hline
\textbf{RF17}   & \vtop{\hbox{\strut O usuário deve ser capaz de editar o resultado do cálculo enzimático}\hbox{\strut antes de salvar no seu experimento}} \\ \hline
\textbf{RF18}   & \vtop{\hbox{\strut O usuário deve ser capaz de salvar o resultado do cálculo enzimático}\hbox{\strut no seu experimento}} \\ \hline
\textbf{RF19}   & \vtop{\hbox{\strut O usuário deve ser capaz de visualizar os resultados e o progresso}\hbox{\strut do seu experimento}} \\ \hline
\textbf{RF20}   & O usuário deve ser capaz de exportar os resultados do seu experimento  \\ \hline
% \end{longtblr}

\end{tabular}%
}
\caption{Descrição dos requisitos funcionais do sistema}
\label{table:requisitos_funcionais}
\end{table}

Segundo \cite{pressman2016engenharia}, os requisitos funcionais, que descrevem as funções e serviços que o software deve oferecer, descrevem como o software deve funcionar, quais entradas devem ser processadas e quais saídas devem ser geradas em resposta a essas entradas. Neste sentido, os requisitos funcionais descritos na \tabref{table:requisitos_funcionais} têm o objetivo de especificar as funcionalidades que o sistema proposto deve oferecer.

\subsection{Casos de uso}\label{ssec:casos_de_uso}

Os Casos de Uso (\textit{Use Cases}) são uma técnica de modelagem de requisitos de software que descrevem as interações entre o usuário e o sistema. Eles ajudam a identificar as funcionalidades e serviços que o software deve oferecer, além de fornecer um meio de comunicação claro entre as equipes de desenvolvimento e os \textit{stakeholders}. A partir disto, foi elaborado um conjunto de casos de uso do sistema, demonstrado na \tabref{table:casos_de_uso}, com o objetivo de centrar os requisitos funcionais descritos na \tabref{table:requisitos_funcionais} e relacionar os mesmos com os tipos de usuários do sistema.

\begin{table}[H]
\centering
\resizebox{\textwidth}{!}{%
\begin{tabular}{|l|l|l|l|}
\hline
\textbf{Código} & \textbf{Descrição} & \textbf{Tipo de usuário} & \textbf{\vtop{\hbox{\strut Requisitos funcionais}\hbox{\strut contemplados}}}\\ \hline
UC01 & Fazer cadastro no sistema & \vtop{\hbox{\strut Usuário comum,}\hbox{\strut Administrador}} & RF01 \\ \hline
UC02 & Fazer \textit{login} no sistema & \vtop{\hbox{\strut Usuário comum,}\hbox{\strut Administrador}} & RF02 \\ \hline
UC03 & Recuperar senha no sistema & \vtop{\hbox{\strut Usuário comum,}\hbox{\strut Administrador}} & RF03 \\ \hline
UC04 & Criar, deletar, visualizar e usar tratamentos & \vtop{\hbox{\strut Usuário comum,}\hbox{\strut Administrador}} & RF04, RF05 e RF06 \\ \hline
UC05 & Criar e deletar enzimas & Administrador & RF07 e RF08 \\ \hline
UC06 & Visualizar e usar enzimas & \vtop{\hbox{\strut Usuário comum,}\hbox{\strut Administrador}} & RF09 \\ \hline
UC07 & Criar, deletar, filtrar e visualizar experimentos & \vtop{\hbox{\strut Usuário comum,}\hbox{\strut Administrador}} &  RF10, RF11, RF12 e RF13 \\ \hline
UC08 & Visualizar detalhes do experimento & \vtop{\hbox{\strut Usuário comum,}\hbox{\strut Administrador}} & RF14 \\ \hline
UC09 & \vtop{\hbox{\strut Inserir dados, editar e salvar o}\hbox{\strut cálculo enzimático no experimento}} & \vtop{\hbox{\strut Usuário comum,}\hbox{\strut Administrador}} & RF15, RF16, RF17 e RF18 \\ \hline
UC10 & Visualizar resultados do experimento & \vtop{\hbox{\strut Usuário comum,}\hbox{\strut Administrador}} & RF19 \\ \hline
UC11 & Compartilhar resultados do experimento & \vtop{\hbox{\strut Usuário comum,}\hbox{\strut Administrador}} & RF20 \\ \hline

\end{tabular}
}
\caption{Casos de Uso do sistema}
\label{table:casos_de_uso}
\end{table}

Uma reprodução visual dos Casos de Uso apontados na \tabref{table:casos_de_uso} e as suas interações com os usuários foi elaborada, na \figref{fig:use-cases-enzitech}. Este diagrama utiliza \ac{uml} — em português: Linguagem de Modelagem Unificada — uma linguagem para visualização, especificação, construção e documentação de artefatos de um software em desenvolvimento, desta forma o Diagrama de \textit{Use Cases} \footnote{Diagrama de \textit{Use Cases}: \url{http://www.dsc.ufcg.edu.br/~jacques/cursos/map/html/uml/diagramas/usecases/usecases.htm}} utilizado abaixo tem o objetivo de auxiliar a comunicação entre os analistas e o cliente, descrevendo um cenário que mostra as funcionalidades do sistema do ponto de vista do usuário. O cliente deve ver neste diagrama as principais funcionalidades de seu sistema.

\begin{figure}[H]
\centering
  \includegraphics[width=\columnwidth]{images/use-cases-enzitech.png}
  \caption{Diagrama de Casos de Uso do sistema.}
  % \acsfont{Fonte: Autoria própria}
  \label{fig:use-cases-enzitech}
\end{figure}

Graças ao diagrama acima, é possível notar, de forma simples, que um usuário administrador possui um privilégio em comparação ao usuário comum: a criação e exclusão das enzimas, este Caso de Uso que ficou restrito à administração do sistema por ser compartilhada entre todos os usuários do Enzitech, podendo causar perdas de dados se sua alteração for realizada por um usuário comum, portando, surgiu a necessidade de diferenciar ambos os tipos de usuários no sistema e realizar estas validações para a disponibilização dessa \textit{feature}. 

\section{Protótipo}
\cite{warfel2009prototyping} destaca que os protótipos de baixa fidelidade são rápidos e baratos de criar, permitindo que os \textit{designers} testem e refinem rapidamente ideias iniciais. Já os protótipos de alta fidelidade, embora mais caros e demorados, podem ajudar a obter \textit{feedback} mais preciso e detalhado sobre o \textit{design} e a usabilidade do software.

Dentre as diversas reuniões do projeto, houve a discussão das funcionalidades para gerar um protótipo de baixa fidelidade, para possíveis refinamentos, antes da elaboração de um protótipo de alta fidelidade, que demanda mais tempo e esforço. Junto desse protótipo, foi idealizado a identidade visual e o nome oficial para o projeto, que passou a se chamar "Enzitech".

\begin{figure}[H]
\centering
  \includegraphics[width=\columnwidth]{images/prototipo_baixa.png}
  \caption{Parte do protótipo final de baixa fidelidade.}
  % \acsfont{Fonte: Autoria própria}
  \label{fig:prototipo_baixa}
\end{figure}

Como é possível ver na \figref{fig:prototipo_baixa}, o protótipo de baixa fidelidade trás o desenho das principais funcionalidades do sistema, além de dar uma ideia do layout final da aplicação.

\begin{figure}[H]
\centering
  \includegraphics[width=\columnwidth]{images/exemplo_prototipo_alta.png}
  \caption{Parte do protótipo final de alta fidelidade.}
  % \acsfont{Fonte: Autoria própria}
  \label{fig:exemplo_prototipo_alta}
\end{figure}

Já nas \figref{fig:exemplo_prototipo_alta} e \figref{fig:prototipo_alta}, é possível verificar mais a fundo todo o \textit{design-system} aplicado, averiguar comportamentos das telas, fluxo de navegação, entre outras vantagens, trazendo uma versão muito mais refinada da aplicação, auxiliando o desenvolvimento e facilitando a visão do sistema para o cliente, onde antes era definido somente por Casos de Uso.

\begin{figure}[H]
\centering
  \includegraphics[width=\columnwidth]{images/prototipo_alta.png}
  \caption{Visão geral do protótipo final de alta fidelidade.}
  % \acsfont{Fonte: Autoria própria}
  \label{fig:prototipo_alta}
\end{figure}


\section{Tecnologias e ferramentas envolvidas}\label{sec:detalhes_tec}
Nesta seção, serão apresentados aspectos como tecnologias utilizadas, os passos necessários para a configuração do ambiente de desenvolvimento, a instalação do Flutter, ferramentas e bibliotecas utilizadas, integrações, gerenciamento de dependências, modelos de dados e o fluxo de trabalho.

\subsection{Instalação do Flutter}\label{ssec:instalacao_flutter}
% Instalação do Flutter e configuração do ambiente de desenvolvimento: Aqui, é importante descrever os passos necessários para instalar o Flutter e configurar o ambiente de desenvolvimento, incluindo a instalação do Android Studio, do emulador Android e da SDK do Android.
Antes de tudo, é necessário elucidar os requisitos para montar um ambiente de desenvolvimento voltado à criação de \acp{app} em Flutter, este \ac{sdk} está disponível nos principais \acp{so} de \acp{pc}, ChromeOS, macOS, Linux e Windows, sendo este último o escolhido como ambiente para o desenvolvimento desta aplicação, o qual será explicado o processo de instalação aqui.

De acordo com o site oficial\footnote{\label{flutter_install}Instalação do Flutter no Windows: \url{https://docs.flutter.dev/get-started/install/windows}}, para instalar e executar o Flutter, o ambiente de desenvolvimento deve atender a estes requisitos mínimos:
\begin{itemize}
   \item Windows 10 ou posterior (64 bits), baseado em x86-64;
   \item 1,64 GB de espaço livre no disco (não inclui espaço em disco para IDE/ferramentas);
   \item Ferramentas no ambiente do Windows:
   \begin{itemize}
     \item Windows PowerShell 5.0 ou mais recente (pré-instalado com o Windows 10);
     \item Git para Windows 2.x, com a opção "Use o Git no prompt de comando do Windows";
   \end{itemize}
 \end{itemize}

 Com todos os requisitos preenchidos, basta seguir o passo a passo detalhado como consta no site oficial\footref{flutter_install}.

\subsection{Instalação do Android Studio}\label{ssec:instalacao_android_studio}
Como consta no tutorial oficial\footref{flutter_install}, a instalação do Android Studio é essencial, porém neste passo, gosto de utilizar o \textit{JetBrains Toolbox App}\footnote{\label{toolbox}\textit{JetBrains Toolbox App}: \url{https://www.jetbrains.com/pt-br/toolbox-app/}}, que faz a instalação e futuras atualizações em poucos cliques. Após instalar o \textit{JetBrains Toolbox App}, basta achar o Android Studio e clicar em instalar.

\begin{figure}[H]
\centering
  \includegraphics{images/toolbox.png}
  \caption{\textit{JetBrains Toolbox App}}
  % \acsfont{Fonte: Autoria própria}
  \label{fig:toolbox}
\end{figure}

Após instalado, é necessário abrir o Android Studio, fazer sua configuração inicial e configurar mais alguns passos, dentre eles, fazer a instalação de um emulador Android para utilizar no desenvolvimento, caso não queira utilizar um dispositivo físico, veja a seguir:
\begin{enumerate}
   \item Habilitar a "aceleração de VM" em sua máquina Windows;
   \item Na tela inicial do Android Studio, clique no ícone \textit{AVD Manager} e selecione \textit{Create Virtual Device…};
   \item Escolha uma definição de dispositivo e selecione "Avançar";
   \item Selecione uma ou mais imagens do sistema para as versões do Android que deseja emular e selecione Avançar. Uma imagem x86 ou x86\_64 é recomendada;
   \item Na aba de Performance do emulador, selecione Hardware - GLES 2.0 para habilitar a aceleração de hardware;
   \item Verifique se tudo está correto e selecione "Concluir";
   \item Por fim, basta executar o emulador e ele iniciará;
 \end{enumerate}

 Outro passo imprescindível é ativar as ferramentas \ac{sdk} necessárias para a compilação do projeto Flutter em um arquivo executável \textsc{.apk} para o Android, basta seguir este passo a passo:
 
 \begin{enumerate}
   \item Na tela inicial do Android Studio, clicar em \textit{File > Settings > Appearance \& Behavior > System Settings > \textit{Plugins}};
   \item Na aba \textit{Marketplace}, busque e instale os \textit{plugins} "Dart" e "Flutter";
   \item Nesta mesma tela, no menu lateral, clique em \textit{Appearance \& Behavior > System Settings > Android \ac{sdk}};
   \item Na aba \textit{\ac{sdk} Tools}, verifique se existem versões instaladas para \textit{Android SDK Build-Tools 34-rc2}, caso não, instale a correspondente à versão do Android escolhida na criação do emulador ou a do seu dispositivo físico;
   \item Ainda na aba \textit{\ac{sdk} Tools}, verifique se existe pelo menos uma versão instalada para \textit{Android SDK Command-line Tools}, caso contrário, instale a última versão disponível;
   \item Ainda na aba \textit{\ac{sdk} Tools}, verifique se o \textit{Android SDK Platform-Tools} está instalado, caso contrário, instale;
   \item Opcional: Ainda na aba \textit{\ac{sdk} Tools}, verifique se o \textit{Intel x86 Emulator Accelerator (HAXM installer)} está instalado, caso contrário, instale (esta ferramenta melhora a performance do emulador caso seu sistema suporte);
 \end{enumerate}
 
\subsection{Instalação do VSCode}\label{ssec:instalacao_vscode}
É possível seguir com o desenvolvimento no Android Studio, porém ele é um software que exige bastante processamento, o que pode "pesar" no sistema, principalmente se utilizado junto de um emulador Android, portanto, existe a possibilidade de utilizar o VSCode\footnote{\label{vscode}VSCode: \url{https://code.visualstudio.com/}} para seguir como software padrão de desenvolvimento, além disso, no VSCode é possível instalar uma grande quantidade de pacotes desenvolvidos pela comunidade que agilizam o desenvolvimento, para utilizar o VSCode basta instalá-lo pela loja do \ac{so}, no caso do Windows, a \textit{Microsoft Store} ou pelo site oficial\footref{vscode}. 

Após instalado e feito a configuração inicial, basta buscar e instalar os \textit{plugins} "Dart" e "Flutter", outros de sua preferência também podem ser instalados. Por fim, basta criar (seguindo o tutorial oficial\footnote{\label{create_app}Escreva seu primeiro aplicativo Flutter: \url{https://docs.flutter.dev/get-started/codelab}}) ou abrir um projeto Flutter já existente e começar a editar seu código.

\subsection{Ferramentas e bibliotecas de suporte}\label{ssec:ferramentas}
% Descrever as ferramentas e bibliotecas de suporte utilizadas no desenvolvimento, tais como o Visual Studio Code, o Dart DevTools, o Flutter DevTools e as bibliotecas de terceiros que podem ser utilizadas para ajudar no desenvolvimento da aplicação.
Além dos \acp{ide}, outras ferramentas foram utilizadas para o desenvolvimento do sistema, nesta subseção abordo brevemente cada uma e sua aplicação.
\subsubsection{Flutter DevTools}\label{sssec:flutter_devtools}
O DevTools\footnote{\label{dev_tools}DevTools: \url{https://docs.flutter.dev/development/tools/devtools/overview}} é um conjunto de ferramentas de desempenho e depuração para Dart e Flutter, vem integrado na instalação do Flutter e executa automaticamente na inicialização de \textit{debugging}, suas principais funções são:

 \begin{enumerate}
   \item Inspecionamento do layout da interface do usuário e o estado de um aplicativo Flutter;
   \item Diagnóstico de problemas de instabilidade no desempenho da interface do usuário em um aplicativo Flutter;
   \item Visualização de informações gerais de \textit{log} e diagnóstico sobre um aplicativo de linha de comando Flutter ou Dart em execução;
   % \item Entre outras funcionalidades;
 \end{enumerate}

\subsubsection{Postman}\label{sssec:postman}
O Postman\footnote{\label{postman}Postman: \url{https://www.postman.com/}} é uma ferramenta de teste de \ac{api} que permite aos desenvolvedores testar, documentar e compartilhar suas APIs. Com o Postman, é possível enviar solicitações \ac{http}, verificar as respostas da \ac{api}, criar scripts de teste automatizados e compartilhar coleções de solicitações com outras pessoas. A ferramenta é amplamente utilizada no desenvolvimento de software e pode ser integrada a outras ferramentas, como o Swagger\footnote{\label{swagger}Swagger: \url{https://swagger.io/}}, para tornar o processo de teste e documentação de \acp{api} mais eficiente.

\subsubsection{\textit{Packages} do pub.dev}\label{sssec:pubdev}
Os \textit{packages} do pub.dev\footnote{\label{pubdev}pub.dev: \url{https://pub.dev/}} são bibliotecas de código-fonte aberto, desenvolvidas por membros da comunidade Flutter, que podem ser utilizadas para implementar recursos em \acp{app} Flutter. Essas bibliotecas fornecem funcionalidades prontas para uso, como a integração com \acp{api}, manipulação de imagens, gerenciamento de estado, entre outras. 

Ao utilizar um \textit{package} do pub.dev, os desenvolvedores podem economizar tempo e esforço na implementação de funcionalidades comuns, além de poderem se beneficiar de contribuições e correções de \textit{bugs} feitas pela comunidade. Os \textit{packages} podem ser facilmente adicionados ao projeto por meio do arquivo \textsc{pubspec.yaml} e instalados por meio do comando \inlinecode{flutter pub get}.


\chapter{Desenvolvimento}

\section{Back-end}

\lipsum[1]

\section{Comunicação por API}

\lipsum[1]

\section{Aplicativo móvel desenvolvido}

% \subsection{Subsection}

\lipsum[1]
\chapter{Análise dos resultados}\label{ch:resultados}
A ideia deste capítulo é trazer uma visão geral sobre todas as funcionalidades desenvolvidas, o fluxo de navegação entre as \textit{features} e como o \ac{app} ficou em sua última versão.

\section{Apresentação do Enzitech}
Após concluído todos os requisitos funcionais, a versão final do Enzitech lida com dois tipos de usuário, Comum ou Administrador. O primeiro deles é referente ao perfil de todos os usuários, pessoas que utilizarão o sistema para criar e calcular seus experimentos. Já o segundo é destinado ao administrador geral de todo o sistema, normalmente uma única pessoa que ficará encarregada de gerir o Enzitech disponibilizando os dados corretamente. 

A diferença entre os dois perfis está na possibilidade de criação de enzimas, funcionalidade restrita ao Administrador, devido a necessidade de ajuste também no \textit{back-end}, ou seja, o Administrador fica responsável por gerir (criar e excluir) enzimas e solicitar a inclusão de novos cálculos para outros tipos, outro detalhe importante é que as enzimas criadas pelo Administrador ficam disponíveis para todos os usuários do sistema, as demais funcionalidades ficam disponíveis para ambos os tipos de usuário. 

Sendo assim, para ter acesso ao sistema, o usuário terá que inserir seus dados de acesso, e-mail e senha, na tela de login (\figref{fig:fluxo_login}). A autenticação é fundamental para que o usuário tenha acesso às funcionalidades do \ac{app}, o login satisfaz o caso de uso UC02.


\begin{figure}[H]
\centering
\includegraphics[width=.3\textwidth]{images/enzitech/splash.pdf}
\includegraphics[width=.3\textwidth]{images/enzitech/login.pdf}
\includegraphics[width=.3\textwidth]{images/enzitech/home.pdf}
\caption{Fluxo de login do Administrador previamente cadastrado.}
\acsfont{Fonte: Aplicativo Enzitech desenvolvido pelo autor}
\label{fig:fluxo_login}
\end{figure}

Na \figref{fig:fluxo_login}, é possível ver o fluxo inicial do \ac{app}, ao abrí-lo, é feita uma verificação na \textit{splashscreen} (primeira imagem da sequência) para determinar se existe usuário logado ou não, caso negativo, o \ac{app} redireciona para a tela de login, onde é possível criar uma conta (caso de uso UC01), recuperar senha (caso de uso UC03) ou logar com suas credencias, após o login, ou, caso o usuário já estivesse logado, o app redireciona para a \textit{homepage}, onde estão todas as funcionalidades disponíveis, a primeira delas é a listagem de experimentos (caso de uso UC07), nesta tela é possível além da listagem, filtrar, excluir ou criar, como será mostrado em breve.

\begin{figure}[H]
\centering
\includegraphics[width=.4\textwidth]{images/enzitech/enzimas.pdf}\hfill
\includegraphics[width=.4\textwidth]{images/enzitech/cria_enzima.pdf}\hfill
\caption{Fluxo de listagem, exclusão e criação de enzimas.}
\acsfont{Fonte: Aplicativo Enzitech desenvolvido pelo autor}
\label{fig:fluxo_enzima}
\end{figure}

Na \figref{fig:fluxo_enzima}, estão as funcionalidades de listagem, exclusão e criação de enzimas, esta última restrita ao administrador, satisfazendo os casos de uso UC05 e UC06. Criar uma enzima no sistema é fundamental para a criação de um experimento.

Abaixo, na \figref{fig:fluxo_tratamento}, está o fluxo de listagem, criação e exclusão dos tratamentos (caso de uso UC04), a criação de um tratamento também é fundamental para a criação de um experimento.

\begin{figure}[H]
\centering
\includegraphics[width=.4\textwidth]{images/enzitech/tratamentos.pdf}\hfill
\includegraphics[width=.4\textwidth]{images/enzitech/cria_tratamento.pdf}\hfill
\caption{Fluxo de listagem, exclusão e criação de tratamentos.}
\acsfont{Fonte: Aplicativo Enzitech desenvolvido pelo autor}
\label{fig:fluxo_tratamento}
\end{figure}

Após criado pelo menos uma enzima e um tratamento, o usuário consegue criar seu primeiro experimento, como mostrado no fluxo da \figref{fig:fluxo_cria_experimento} abaixo.

\begin{figure}[p]
  \centering
  \includegraphics[width=.3\textwidth]{images/enzitech/home_2.pdf}
  \includegraphics[width=.3\textwidth]{images/enzitech/cria_exp_1.pdf}
  \includegraphics[width=.3\textwidth]{images/enzitech/cria_exp_2.pdf}

  \vspace{1cm}

  \includegraphics[width=.3\textwidth]{images/enzitech/cria_exp_3.pdf}
  \includegraphics[width=.3\textwidth]{images/enzitech/cria_exp_4.pdf}
  \includegraphics[width=.3\textwidth]{images/enzitech/cria_exp_5.pdf}

  \caption{Fluxo de criação de um experimento.}
  \label{fig:fluxo_cria_experimento}
  \acsfont{Fonte: Aplicativo Enzitech desenvolvido pelo autor}
  
\end{figure}


É possível ver na primeira imagem, na tela de experimentos, o filtro de "experimentos concluídos" aplicado, seguindo o fluxo, na segunda imagem é solicitada as informações de identificação do experimento, nome e descrição, a seguir, os tratamentos daquele experimento e quantas repetições serão realizadas, logo após, surge o seletor de enzimas, na quarta imagem (terceira etapa do processo de criação de experimento), nele é possível escolher quais enzimas farão parte do experimento, logo após surge a última etapa para o preenchimento dos valores variáveis daquele experimento, após criado (concluindo o caso de uso UC07), o usuário é levado para a funcionalidade de visualizar o experimento, explicado na \figref{fig:fluxo_experimento_detalhado} a seguir.

\begin{figure}[p]
  \centering
  \includegraphics[width=.3\textwidth]{images/enzitech/exp_detalhado.pdf}
  \includegraphics[width=.3\textwidth]{images/enzitech/calculo_1.pdf}
  \includegraphics[width=.3\textwidth]{images/enzitech/calculo_2.pdf}

  \vspace{1cm}

  \includegraphics[width=.3\textwidth]{images/enzitech/calculo_3.pdf}
  \includegraphics[width=.3\textwidth]{images/enzitech/calculo_4.pdf}
  \includegraphics[width=.3\textwidth]{images/enzitech/exp_detalhado2.pdf}

  \caption{Fluxo de detalhamento e cálculo enzimático de um experimento.}
  \label{fig:fluxo_experimento_detalhado}
  \acsfont{Fonte: Aplicativo Enzitech desenvolvido pelo autor}
  
\end{figure}

Após o experimento criado, o usuário pode ver seus detalhes, como a quantidade de enzimas, tratamentos e repetições, seu progresso (caso de uso UC08) e a possibilidade de excluí-lo, além disso, o acesso à outras duas funcionalidades, a de cálculo enzimático, para inserção de dados no experimento, e a de resultados, para a visualização e compartilhamento desses dados.

Acompanhando a \figref{fig:fluxo_experimento_detalhado}, para o preenchimento do experimento é necessário escolher um tratamento e uma enzima, assim, gerando um conjunto de informações para prosseguir com a inserção dos valores em suas respectivas repetições.

Após todos os valores inseridos, o usuário clica em carregar e é levado para uma tela de resultados deste cálculo e a média (quarta e quinta imagem), nesta tela, é possível ver os resultados e receber um \textit{feedback} sobre a discrepância dos valores em comparação com a média de todos os resultados das repetições, assim, sendo possível perceber algum valor que pode estar inserido incorretamente, resultando em dados errôneos, desta forma, o usuário pode recalcular, corrigindo com novos valores ou prosseguir, salvando e saindo da tela de cálculo (caso de uso UC09).

Em seguida, quando um experimento já tem dados suficientes (progresso maior ou igual a 1\%), é possível ver e compartilhar os resultados obtidos (casos de uso UC10 e UC11), mostrados a seguir na \figref{fig:fluxo_resultados}.

\begin{figure}[p]
  \centering
  \includegraphics[width=.3\textwidth]{images/enzitech/exp_concluido.pdf}
  \includegraphics[width=.3\textwidth]{images/enzitech/exp_detalhado3.pdf}
  \includegraphics[width=.3\textwidth]{images/enzitech/resultados_salvar.pdf}

  \vspace{1cm}

  \includegraphics[width=.3\textwidth]{images/enzitech/resultados_salvar_downloads.pdf}
  \includegraphics[width=.3\textwidth]{images/enzitech/resultados_salvar_planilha_min.pdf}
  \includegraphics[width=.3\textwidth]{images/enzitech/compartilhar.pdf}

  \caption{Fluxo de visualização e compartilhamento de resultados de um experimento.}
  \label{fig:fluxo_resultados}
  \acsfont{Fonte: Aplicativo Enzitech desenvolvido pelo autor}
  
\end{figure}

As duas primeiras imagens do fluxo da \figref{fig:fluxo_resultados} são de experimentos diferentes, a primeira é de um que já foi concluído, nele é possível ver que o usuário tem a ação de cálculo enzimático bloqueado, já na segunda imagem é o experimento que seguirá o fluxo de visualização e compartilhamento dos resultados aqui, este experimento possui um tratamento, duas repetições, e cinco enzimas cadastradas.

Ao entrar nos resultados do experimento, o usuário consegue visualizar de forma organizada todos os dados preenchidos como uma listagem com tabelas para cada combinação de enzima e tratamento feita, nesta tela, é possível salvar o resultado em uma planilha Excel no formato \textsc{nome-do-experimento.xlsx} no armazenamento do dispositivo, como é possível ver na terceira, quarta e quinta imagem do fluxo da \figref{fig:fluxo_resultados}, a planilha é montada seguindo o mesmo padrão, para cada enzima é criado uma página, e dentro de cada página os resultados são montados com suas repetições para cada tratamento do experimento (caso de uso UC10).

Além disso, o usuário pode compartilhar diretamente o arquivo para qualquer \ac{app} externo que suporte esta ação, ao compartilhar um experimento, o salvamento dele também é realizado (caso de uso UC11).

Por fim, o usuário tem acesso à uma tela de configurações na \textit{home} do \ac{app} (\figref{fig:configuracoes_app}), nela é possível ter acesso às seguintes funcionalidades adicionais: informações do \ac{app}, seus dados de login, uma configuração para a ativação e desativação do \textit{AlertDialog} para confirmação da exclusão de itens no \ac{app}, informação da quantidade de resultados de experimentos salvos localmente, informações sobre ambiente e versão do \ac{app} e a opção de deslogar do sistema. Além, disso, como mostrado na última imagem, o \ac{app} tem uma funcionalidade que informa quando o aplicativo fica sem acesso ao servidor, seja por falha no servidor ou por problemas de conexão com a internet. Por fim, o \ac{app} também contempla a funcionalidade de armazenar e consumir dados em \textit{cache} quando não há conexão com a internet disponível, tornando possível a visualização de algumas informações resumidas.

\begin{figure}[H]
\centering
\includegraphics[width=.3\textwidth]{images/enzitech/config.pdf}\hfill
\includegraphics[width=.3\textwidth]{images/enzitech/config2.pdf}\hfill
\includegraphics[width=.3\textwidth]{images/enzitech/offline.pdf}\hfill
\caption{Tela de configurações do \ac{app} do Enzitech.}
  \acsfont{Fonte: Aplicativo Enzitech desenvolvido pelo autor}
\label{fig:configuracoes_app}
\end{figure}

\chapter{Conclusão}
[EM DESENVOLVIMENTO até 31/03/23]

\section{Trabalhos futuros}
[EM DESENVOLVIMENTO até 31/03/23]

% \lipsum[1]

% \section{Section}

% \lipsum[1]

% \subsection{Subsection}

% \lipsum[1]

% References

\begin{references}
  \bibliography{references}
\end{references}

% Appendix

\theappendix\label{appendix}
\chapter{Códigos Flutter}\label{ch:codigosFlutter}

\section{Código da tela \textit{Home Page} desenvolvida na Terceira Sprint}\label{code:home-sprint3}
\begin{code}
% \captionof{listing}{Home Page}
\begin{minted}{dart}
// imports suprimidos

class HomePage extends StatefulWidget {
  const HomePage({Key? key}) : super(key: key);

  @override
  State<HomePage> createState() => _HomePageState();
}

class _HomePageState extends State<HomePage> {
  late final HomeController controller;
  late final ExperimentsController experimentsController;
  late final TreatmentsController treatmentsController;
  late final AccountController accountController;

  late List<Widget> _fragments;

  @override
  void initState() {
    super.initState();
    controller = context.read<HomeController>();
    experimentsController = context.read<ExperimentsController>();
    treatmentsController = context.read<TreatmentsController>();
    accountController = context.read<AccountController>();
    initFragements();
    if (mounted) {
      Future.delayed(Duration.zero, () async {
        await experimentsController.loadExperiments();
        await treatmentsController.loadTreatments();
        await accountController.loadAccount();
      });
      controller.addListener(() {
        if (controller.state == HomeState.error) {
          ScaffoldMessenger.of(context).showSnackBar(
            SnackBar(
              content: Text(controller.failure!.message),
            ),
          );
        }
      });
    }
  }

  initFragements() {
    _fragments = [
      ExperimentsPage(
        homeController: controller,
      ),
      TreatmentsPage(
        homeController: controller,
      ),
      AccountPage(
        homeController: controller,
      ),
    ];
  }

  Widget? get dealWithFloatingActionButton {
    if (controller.fragmentIndex == 0) {
      return FloatingActionButton.extended(
        onPressed: () {
          Navigator.pushNamed(
            context,
            RouteGenerator.createExperiment,
          );
        },
        label: Text(
          "Cadastrar\nexperimento",
          style: TextStyles.buttonBackground,
        ),
        icon: const Icon(
          PhosphorIcons.pencilLine,
          color: AppColors.white,
          size: 30,
        ),
      );
    }

    if (controller.fragmentIndex == 1) {
      return FloatingActionButton.extended(
        onPressed: () {
          Navigator.pushNamed(
            context,
            RouteGenerator.createTreatment,
          );
        },
        label: Text(
          "Cadastrar\ntratamento",
          style: TextStyles.buttonBackground,
        ),
        icon: const Icon(
          PhosphorIcons.pencilLine,
          color: AppColors.white,
          size: 30,
        ),
      );
    }

    return null;
  }

  @override
  Widget build(BuildContext context) {
    final controller = context.watch<HomeController>();

    return Scaffold(
      appBar: AppBar(
        title: SvgPicture.asset(
          'assets/images/logo.svg',
          fit: BoxFit.contain,
          alignment: Alignment.center,
        ),
      ),
      body: ChangeNotifierProvider(
        create: (BuildContext context) {},
        child: _fragments[controller.fragmentIndex],
      ),
      floatingActionButton: dealWithFloatingActionButton,
      bottomNavigationBar: BottomNavigationBar(
        items: const [
          BottomNavigationBarItem(
            icon: Icon(PhosphorIcons.flask),
            label: 'Experimentos',
          ),
          BottomNavigationBarItem(
            icon: Icon(PhosphorIcons.testTube),
            label: 'Tratamentos',
          ),
          BottomNavigationBarItem(
            icon: Icon(PhosphorIcons.userCircleGear),
            label: 'Conta',
          ),
        ],
        currentIndex: controller.fragmentIndex,
        selectedItemColor: AppColors.white,
        backgroundColor: AppColors.primary,
        onTap: (index) => controller.setFragmentIndex(index),
      ),
    );
  }
}
\end{minted}
\end{code}

\section{Trecho de código para realizar a checagem de autenticação e pré-carregamento dos dados desenvolvido na Sprint 6}\label{code:check-auth}
\begin{code}
% \captionof{listing}{Trecho de código para realizar a checagem de autenticação e pré-carregamento dos dados}
% \label{code:dart-code}
\begin{minted}{dart}
_checkAuth() async {
    Future.delayed(Duration.zero).then((_) async {
    String token = await getToken() ?? '';

    if (!mounted) return;

    if (token.isEmpty) {
        Navigator.pushReplacementNamed(context, RouteGenerator.auth);
    } else {
        await Provider.of<HomeViewmodel>(context, listen: false)
        .getContent()
        .then((value) =>
            Navigator.pushReplacementNamed(
                context, 
                RouteGenerator.home,
            ),
        );
      }
    });
  }
\end{minted}
\end{code}
\chapter{Testes}\label{ch:testes}

\section{Trecho de código do teste de unidade para enzima}\label{code:teste-unidade-enzima}
\begin{code}
% \captionof{listing}{Trecho de código para realizar a checagem de autenticação e pré-carregamento dos dados}
% \label{code:dart-code}
\begin{minted}{dart}
main() {
  test(
    "Should not return null on Enzyme entity",
    () {
      EnzymeEntity enzyme = EnzymeEntity(
        id: "1",
        name: "Enzima",
        type: "Tipo",
        formula: "a+b=c",
        variableA: 0.52,
        variableB: 0.32,
      );

      expect(enzyme, isNotNull);
    },
  );
}
\end{minted}
\end{code}

\end{document}
