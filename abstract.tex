Given the projected growth of the world population, Soil Quality (SQ) is one of the fundamental factors for increasing agricultural productivity and reducing global food shortages. In this context, it is essential to understand SQ, its physical, chemical and biological characteristics, as well as the interactions between these factors, allowing the development of more efficient and sustainable soil management techniques. Several methods are used to evaluate SQ, among them, the calculation of soil Enzyme Activities (EAs) is an important tool for evaluating soil quality and health, since EAs are directly related to the quantity and diversity of microorganisms. present in the soil. Therefore, the availability of accurate information and also of computer systems to calculate EAs and to assist in decision making is of fundamental importance. Therefore, this work proposes a system for managing experiments and analysis results of soil EAs. This approach was implemented through concepts such as agile methods and clean architecture, initially covering the calculation of the enzymes acid and alkaline phosphatase, $\beta$-glucosidase, urease and protease. So, after analyzing the feasibility and specifying the product requirements, together with those involved in the development of this software, a prototype was obtained, from which the solution was designed. Therefore, this tool will help researchers, laboratories and professionals in the field, so that they can carry out scientifically based soil experiments in a practical and precise way, improving management practices.

\begin{keywords}
Soil quality, Extracellular enzymes, Software Engineering, Mobile Development, Agile Methodologies, Clean Architecture, Flutter.
\end{keywords}