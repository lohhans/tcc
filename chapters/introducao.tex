\chapter{Introdução}
\label{chp:introduction}
A eficiente e incessante evolução da tecnologia traz consigo a grande oportunidade — como também a necessidade — de tornar as tarefas cotidianas dos seres humanos cada vez mais triviais e informatizadas, ou seja, realizar uma tarefa de forma que exija o menor esforço possível, dessa forma, algo que requisitaria tempo e atenção de alguém vem a se tornar uma tarefa simples para um computador.

As aplicações móveis, habitualmente conhecidas por \acp{app}, são exemplos claros e amplamente difundidos atualmente, visto que cada vez mais tarefas, desde o envio de mensagens até o pagamento de contas, podem ser realizadas nas palmas das nossas mãos, basicamente utilizando um \textit{smartphone} com acesso à Internet. O que há pouco mais de uma década era visualizado como uma atividade exclusiva tecnologicamente de um \ac{pc}, fosse através de uma aplicação \textit{web} ou \textit{desktop}, hoje é facilmente feito por soluções adaptadas ou similares para as telas dos celulares, tornando-se uma atividade corriqueira na vida da maioria das pessoas.

[EM DESENVOLVIMENTO até 31/03/23]

 % — Falar sobre o uso das planilhas para o publico especifico de agronomia da \ac{ufape}, e desenrolar a solução pensada, que foi o app.

\section[Motivação]{Motivação}
 — Apresentar e discorrer sobre a motivação do projeto em si. — 
[EM DESENVOLVIMENTO até 31/03/23]
 
\section{Objetivo}\label{sec:objetivo}
Este trabalho tem como objetivo detalhar a solução e o processo de desenvolvimento de uma aplicação para dispositivos móveis que informatize o método de fazer experimentos e análises do solo, aplicativo este desenvolvido usando padrões arquiteturais limpos, escaláveis, manuteníveis e testáveis.
[EM DESENVOLVIMENTO até 31/03/23]

\section{Visão geral da proposta}
 — Apresentar e discorrer sobre o objetivo do projeto em si. — 
[EM DESENVOLVIMENTO até 31/03/23]

\section{Organização do texto}
 — Apresentar e discorrer sobre a 
 organização do texto do projeto. — 
[EM DESENVOLVIMENTO até 31/03/23]