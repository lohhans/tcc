Given the projection of world population growth, Soil Quality (SQ) is one of the key factors for increasing agricultural productivity and reducing global food shortages. In this context, it is essential to understand SQ, its physical, chemical and biological characteristics, as well as the interactions between these factors, allowing the development of more efficient and sustainable soil management techniques. Several methods are used to evaluate the SQ, among them, the calculation of Enzymatic Activities (AEs) of the soil is an important tool to evaluate the quality and health of the soil, since the AEs is directly related to the quantity and diversity of microorganisms present in the soil. In this way, the availability of accurate information and also of computational systems for calculating the AEs and to aid in decision-making is of fundamental importance. Thus, this work proposes a system for managing experiments and results of soil AEs analyses. This approach was implemented through concepts such as agile methods and clean architecture, initially contemplating the calculation of acid and alkaline phosphatase enzymes, $\beta$-glucosidase, urease and protease. This tool helps researchers, laboratories and professionals in the area, who can carry out scientifically based soil experiments in a practical and precise way, improving management practices, identifying contamination problems, treatment and remediation.

\begin{keywords}
Calculation of soil enzymatic activities, Software Engineering, Mobile Development, Agile Methodologies, Clean Architecture, Flutter.
\end{keywords}