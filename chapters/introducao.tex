\chapter{Introdução}
\label{chp:introduction}
A eficiente e incessante evolução da tecnologia traz consigo a grande oportunidade — como também a necessidade — de tornar as tarefas cotidianas dos seres humanos cada vez mais triviais e informatizadas, ou seja, realizar uma tarefa de forma que exija o menor esforço possível, dessa forma, algo que requisitaria tempo e atenção de alguém vem a se tornar uma tarefa simples para um computador.

As aplicações móveis, habitualmente conhecidas por \acp{app}, são exemplos claros e amplamente difundidos atualmente, visto que cada vez mais tarefas, desde o envio de mensagens até o pagamento de contas, podem ser realizadas nas palmas das nossas mãos, basicamente utilizando um \textit{smartphone} com acesso à Internet. O que há pouco mais de uma década era visualizado como uma atividade exclusiva tecnologicamente de um \ac{pc}, fosse através de uma aplicação \textit{web} ou \textit{desktop}, hoje é facilmente feito por soluções adaptadas ou similares para as telas dos celulares, tornando-se uma atividade corriqueira na vida da maioria das pessoas.

Em outra esfera, a análise de atividades enzimáticas do solo é um importante indicador da qualidade e saúde do solo, pois essas atividades estão diretamente relacionadas com a disponibilidade de nutrientes e a capacidade do solo em sustentar a vida vegetal. Porém, as técnicas convencionais para a medição dessas atividades são trabalhosas, caras e requerem equipamentos sofisticados.

Segundo \cite{nascimento2017impacto}, o uso de tecnologias nas instituições de pesquisa científica em agronomia tem sido amplamente adotado nas últimas décadas, com o objetivo de aumentar a eficiência e a precisão dos experimentos, bem como de facilitar a coleta e análise de dados. A implementação de sistemas informatizados e a utilização de equipamentos avançados de análise, como espectrômetros e microscópios eletrônicos, são exemplos dessas tecnologias que vêm sendo empregadas em estudos agronômicos.

Nesse contexto, o desenvolvimento de um aplicativo móvel para a realização de experimentos de análise e cálculo de atividades enzimáticas do solo pode ser uma solução eficaz e prática para essa demanda. O objetivo deste trabalho é apresentar o desenvolvimento de um aplicativo móvel que permite a realização destes experimentos, com o intuito de facilitar e tornar mais acessível a análise dessas atividades, buscando harmonizar o crescimento econômico, a equidade social e a preservação ambiental, visando à conservação do meio ambiente e à garantia do bem-estar humano a longo prazo.

\section[Motivação]{Motivação}
Os experimentos do solo com cálculo de atividades enzimáticas são uma técnica utilizada em estudos de ciência do solo para avaliar a saúde e qualidade do solo, estes experimentos apresentam alguns desafios e lacunas, como a dificuldade na padronização dos métodos de análise, a falta de informações sobre a relação entre as atividades enzimáticas e a fertilidade do solo e a dificuldade em avaliar a dinâmica temporal das atividades enzimáticas. Além disso, há pouco conhecimento sobre as atividades enzimáticas em diferentes tipos de solos e em diferentes regiões do mundo, e a identificação de microrganismos e enzimas específicas pode ser desafiadora. Esses desafios podem afetar a interpretação dos resultados dos experimentos e limitar a aplicação das técnicas em diferentes contextos \cite{tabatabai1994soil}.

Este tema é importante para a comunidade acadêmica e para a sociedade em geral, pois permite avaliar a saúde e qualidade do solo, identificar microrganismos e enzimas benéficos para o solo e para as plantas, promovendo práticas agrícolas mais sustentáveis e eficientes, além de monitorar a qualidade do solo em áreas urbanas e industriais. A pesquisa e desenvolvimento do sistema proposto para essa área contribui para o desenvolvimento sustentável e para a conservação do meio ambiente.
 
\section{Objetivo Geral}\label{sec:objetivo_geral}

Este trabalho tem como objetivo detalhar a solução e o processo de desenvolvimento de uma aplicação para dispositivos móveis que informatize o método de fazer experimentos e análises do solo, aplicativo este desenvolvido usando padrões arquiteturais limpos, escaláveis, manuteníveis e testáveis.

\section{Objetivos específicos}\label{sec:objetivo_específico}
Com base no objetivo geral, correspondem os objetivos específicos indicados a seguir:

\begin{itemize}
    \item Implementar um sistema, acessível via aplicativo móvel, que permite criar e gerenciar experimentos que realizem cálculos das atividades enzimáticas do solo;
    \item Gerar resultados e planilhas de forma automática;
    \item Possibilitar a análise dos resultados;
    \item Proporcionar uma experiência fluida de navegação no aplicativo, ao utilizar boas práticas de desenvolvimento de \textit{software}.
\end{itemize}

\section{Organização do texto}

O presente trabalho foi assim constituído:
\begin{itemize}
    \item O Capítulo \ref{ch:teoria} discorre sobre conceitos básicos dos principais fundamentos adotados neste trabalho e relacionados ao tema científico, em que são abordado os temas de atividades enzimáticas do solo, dispositivos móveis, desenvolvimento de aplicativos móveis e arquitetura limpa;
    \item No Capítulo \ref{ch:metodos} é realizada uma apresentação dos métodos utilizados no desenvolvimento da solução, bem como os requisitos do sistema, tecnologias e ferramentas envolvidas;
    \item No Capítulo \ref{ch:proposta}, são apresentados os processos do desenvolvimento, arquiteturas utilizadas, informações sobre os testes e implantação do aplicativo e do sistema em si;
    \item No Capítulo \ref{ch:resultados}, é apresentado a versão final do aplicativo desenvolvido, o fluxo de navegação e como todo o sistema se comporta;
    \item No Capítulo \ref{ch:conclusao}, são apresentadas as considerações finais, principais contribuições, dificuldades encontradas e trabalhos futuros.
\end{itemize}
